\documentclass[12pt]{article}
\usepackage{Environments}
\usepackage{Packages}
\title{}
\author{Josh Morales}
\date{\today}
\setlength{\headheight}{15pt}
\begin{document}
\pagestyle{fancy}
\fancyhead[L]{Numerical Analysis HW 6}
\fancyhead[R]{Coyne, Dedvukaj, Gao, Karabushin, Lin, Morales}
\begin{center}
\textbf{\Large Homework 6} \\
\text{Due date}: \today
\end{center}
\begin{enumerate}[leftmargin=2em]
    %Question 1
    \item 
    To solve the system of linear equations, we can create the following matrix:
    \[
    \left[ \begin{array}{ccc|c}
        3 & 2 & -1 & 7 \\
        5 & 3 & 2 & 4 \\
        -1 & 1 & -3 & -1
    \end{array} \right]
    \xrightarrow{R_1 \leftrightarrow R_2}
    \left[ \begin{array}{ccc|c}
        5 & 3 & 2 & 4 \\
        3 & 2 & -1 & 7 \\
        -1 & 1 & -3 & -1
    \end{array} \right]
    \xrightarrow{R_2 + 3R_3}
    \left[ \begin{array}{ccc|c}
        5 & 3 & 2 & 4 \\
        0 & 5 & -10 & 4 \\
        -1 & 1 & -3 & -1
    \end{array} \right]
    \xrightarrow{R_3 + \frac{1}{5}R_1}
    \]
    \[
    \left[ \begin{array}{ccc|c}
        5 & 3 & 2 & 4 \\
        0 & 5 & -10 & 4 \\
        -1 & \frac{8}{5} & \frac{-13}{5} & \frac{-1}{5}
    \end{array} \right]
    \xrightarrow{R_3 - \frac{8}{25}R_2}
    \left[ \begin{array}{ccc|c}
        5 & 3 & 2 & 4 \\
        0 & 0 & -10 & 4 \\
        0 & 0 & \frac{3}{5} & \frac{-37}{25}
    \end{array} \right]
    \]
    Which gives us our partially pivoted matrix. We can now solve the system of equations! 
    \[\frac{3}{5}x_3 = \frac{-37}{25}\]
    \[x_3 = \frac{-37}{25}\times\frac{5}{3} = \frac{-37}{15}\]
    \[5x_2-10 \times \frac{-37}{15} = 4\]
    \[75x_2 + 370 = 60\]
    \[75x_2 + 370 = -310\]
    \[x_2 = \frac{-62}{15}\]
    \[5x_1 + 3 \times \frac{-62}{15} + 2 \times \frac{-37}{15} = 4\]
    \[75x_1 + 3 \times -62 + 2 \times -37 = 60\]
    \[75x_1 - 260 = 60\]
    \[75x_1 = 320\]
    \[x_1 = \frac{64}{15}\]
    %Question 2
    \item

    %Question 3
    
    %Question 4
    \item 


    %Question 5
    \item
    \begin{enumerate}[leftmargin=!]
        %5a
        \item 
        \begin{proof}
            By the proposition, there exists $C, C_{\infty}\in \RR_{>0}$ such that
            \[||x||\leq C_{\infty} ||x||_{\infty}\leq C_{\infty} C ||x||'\]
            For all $x\in \RR^{n}$. Similarly, there exists $C',C'_{\infty}\in \RR_{>0}$ such that 
            \[||x||' \leq C'_{\infty} ||x||_{\infty} \leq C'_{\infty} C' ||x||\]
            For all $x\in \RR^n$. Therefore, letting $D=C_{\infty}C$ and $D'=  C'_{\infty} C'$ completes the proof.
        \end{proof}
        %5b
        \item 
        \begin{proposition}\label{prop:subaddsqrt}
            Let $a,b\in \RR_{\geq 0}$. Then
            \[\sqrt{a+b} \leq \sqrt{a}+\sqrt{b}.\]
        \end{proposition}
        \begin{proof}[Proof of Proposition~\ref{prop:subaddsqrt}]
            Note that since $a$ and $b$ are non-negative, we have that
            \[a+b \leq a+b+2\sqrt{a}\sqrt{b} = {(\sqrt{a}+\sqrt{b})}^2.\]
            Taking the square root on both sides completes the proof.
        \end{proof}

        Note that by Proposition~\ref{prop:subaddsqrt}, we have the following for any $x\in \RR^2$:
        \[||x||_{2}=\sqrt{x_1^2+x_2^2} \leq \sqrt{x_1^2}+\sqrt{x_2^2} = |x_1|+|x_2| = ||x||_{1}.\]
        Furthermore, by the Cauchy-Schwarz-Bunyakovsky Inequality, we have that
        \[||x||_{1} = |x_1|+|x_2| = 1\cdot |x_1|+1\cdot|x_2| \leq \sqrt{1+1}\sqrt{|x_1|^2+|x_2|^2} = \sqrt{2}||x||_{2}.\]
        Therefore, $C_1=1$ and $C_2 = \sqrt{2}$ gives us the desired result.
    \end{enumerate}

    %Question 6
    \item
    \begin{enumerate}[leftmargin=!]
        %6a
        \item 

        %6b
        \item
    \end{enumerate}
    
\end{enumerate}
\end{document}