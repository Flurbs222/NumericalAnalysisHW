\documentclass[12pt]{article}
\usepackage{Environments}
\usepackage{Packages}
\title{}
\author{Josh Morales}
\date{\today}
\setlength{\headheight}{15pt}
\begin{document}
\pagestyle{fancy}
\fancyhead[L]{Numerical Analysis HW 6}
\fancyhead[R]{Coyne, Dedvukaj, Gao, Karabushin, Lin, Morales}
\begin{center}
\textbf{\Large Homework 6} \\
\text{Due date}: \today
\end{center}
\begin{enumerate}[leftmargin=2em]
    %Question 1
    \item 
    To solve the system of linear equations, we can create the following matrix:
    \[
    \left[ \begin{array}{ccc|c}
        3 & 2 & -1 & 7 \\
        5 & 3 & 2 & 4 \\
        -1 & 1 & -3 & -1
    \end{array} \right]
    \xrightarrow{R_1 \leftrightarrow R_2}
    \left[ \begin{array}{ccc|c}
        5 & 3 & 2 & 4 \\
        3 & 2 & -1 & 7 \\
        -1 & 1 & -3 & -1
    \end{array} \right]
    \xrightarrow{R_2 + 3R_3}
    \left[ \begin{array}{ccc|c}
        5 & 3 & 2 & 4 \\
        0 & 5 & -10 & 4 \\
        -1 & 1 & -3 & -1
    \end{array} \right]
    \xrightarrow{R_3 + \frac{1}{5}R_1}
    \]
    \[
    \left[ \begin{array}{ccc|c}
        5 & 3 & 2 & 4 \\
        0 & 5 & -10 & 4 \\
        -1 & \frac{8}{5} & \frac{-13}{5} & \frac{-1}{5}
    \end{array} \right]
    \xrightarrow{R_3 - \frac{8}{25}R_2}
    \left[ \begin{array}{ccc|c}
        5 & 3 & 2 & 4 \\
        0 & 5 & -10 & 4 \\
        0 & 0 & \frac{3}{5} & \frac{-37}{25}
    \end{array} \right]
    \]
    Which gives us our partially pivoted matrix. We can now solve the system of equations! 
    \[\frac{3}{5}x_3 = \frac{-37}{25}\]
    \[x_3 = \frac{-37}{25}\times\frac{5}{3} = \frac{-37}{15}\]
    \[5x_2-10 \times \frac{-37}{15} = 4\]
    \[75x_2 + 370 = 60\]
    \[75x_2 + 370 = -310\]
    \[x_2 = \frac{-62}{15}\]
    \[5x_1 + 3 \times \frac{-62}{15} + 2 \times \frac{-37}{15} = 4\]
    \[75x_1 + 3 \times -62 + 2 \times -37 = 60\]
    \[75x_1 - 260 = 60\]
    \[75x_1 = 320\]
    \[x_1 = \frac{64}{15}\]
    %Question 2
    \item
    To solve the system of linear equations using the inverse matrix, we can do the following: 
    \[ 
    \left[ \begin{array}{ccc|ccc}
        3 & 2 & -1 & 1 & 0 & 0  \\
        5 & 3 & 2 & 0 & 1 & 0 \\
        -1 & 1 & -3 & 0 & 0 & 1
    \end{array}\right]
    \xrightarrow{R_1 \leftrightarrow R_3}
    \left[ \begin{array}{ccc|ccc}
        -1 & 1 & -3 & 0 & 0 & 1  \\
        5 & 3 & 2 & 0 & 1 & 0 \\
        3 & 2 & -1 & 1 & 0 & 0
    \end{array}\right]
    \xrightarrow{R_1 \leftarrow (-1)R_1}
    \left[ \begin{array}{ccc|ccc}
        1 & -1 & 3 & 0 & 0 & -1  \\
        5 & 3 & 2 & 0 & 1 & 0 \\
        3 & 2 & -1 & 1 & 0 & 0
    \end{array}\right]
    \]
    \[
    \xrightarrow{\substack{R_2 \leftarrow R_2 - \frac{1}{5}R_1 \\ R_3 \leftarrow R_3 - 3R_1}}
    \left[ \begin{array}{ccc|ccc}
        1 & -1 & 3 & 0 & 0 & -1  \\
        0 & 1 & -13 & 0 & 1 & 5 \\
        0 & 5 & -10 & 1 & 0 & 3
    \end{array}\right]
    \xrightarrow{\substack{R_2 \leftarrow \frac{1}{8}R_2 \\ R_3 \leftarrow \frac{1}{5}R_3}}
    \left[ \begin{array}{ccc|ccc}
        1 & -1 & 3 & 0 & 0 & -1  \\
        0 & 1 & \frac{-13}{8} & 0 & \frac{1}{8} & \frac{5}{8} \\
        0 & 1 & -2 & \frac{1}{5} & 0 & \frac{3}{5}
    \end{array}\right]
    \xrightarrow{R_3 \leftarrow R_3 - R_2}
    \]
    \[ 
    \left[ \begin{array}{ccc|ccc}
        1 & -1 & 3 & 0 & 0 & -1  \\
        0 & 1 & \frac{-13}{8} & 0 & \frac{1}{8} & \frac{5}{8} \\
        0 & 1 & \frac{-3}{8} & \frac{1}{5} & \frac{-1}{8} & \frac{-1}{40}
    \end{array}\right]
    \xrightarrow{R_3 \leftarrow \frac{-8}{3}R_3}
    \left[ \begin{array}{ccc|ccc}
        1 & -1 & 3 & 0 & 0 & -1  \\
        0 & 1 & 0 & \frac{-13}{15} & \frac{2}{3} & \frac{11}{5} \\
        0 & 0 & 1 & \frac{-8}{15} & \frac{1}{3} & \frac{1}{15}
    \end{array}\right]
    \]
    \[
    \xrightarrow{\substack{R_1 \leftarrow R_1 + R_2 \\ R_1 \leftarrow R_1 - 3R_3}}
    \left[ \begin{array}{ccc|ccc}
        1 & 0 & 0 & \frac{11}{15} & \frac{-1}{3} & \frac{-7}{15}  \\
        0 & 1 & 0 & \frac{-13}{15} & \frac{2}{3} & \frac{11}{5} \\
        0 & 0 & 1 & \frac{-8}{15} & \frac{1}{3} & \frac{1}{15}
    \end{array}\right]
    \]
    Therefore, 
    \[ A^{-1} =
    \left[ \begin{array}{ccc}
        \frac{11}{15} & \frac{-1}{3} & \frac{-7}{15} \\
        \frac{-13}{15} & \frac{2}{3} & \frac{11}{15} \\
        \frac{-8}{15} & \frac{1}{3} & \frac{1}{15} 
    \end{array}\right]
        \] 
    Let
    \[ b =
    \left[ \begin{array}{c}
        7 \\
        4 \\
        1
    \end{array}\right]
        \]
    then 
    \[ 
        x = A^{-1}b = 
        \left[ \begin{array}{ccc}
            \frac{11}{15} & \frac{-1}{3} & \frac{-7}{15} \\
            \frac{-13}{15} & \frac{2}{3} & \frac{11}{15} \\
            \frac{-8}{15} & \frac{1}{3} & \frac{1}{15} 
        \end{array}\right]
        \left[ \begin{array}{c}
            7 \\
            4 \\
            1
        \end{array}\right]
        = 
        \left[ \begin{array}{c}
            \frac{64}{15} \\
            \frac{-62}{15} \\
            \frac{-37}{15}
        \end{array}\right]
    \].
    Therefore, 
    \[x_1 = \frac{64}{15}, x_2 = \frac{-62}{15}, x_3 = \frac{-37}{15}\]

    %Question 3
    \item 
    26 for Gaussian. 43 for matrix inversion. There are significantly less operations for Gaussian elimination therefore making it the more efficient method.
    %Question 4
    \item 


    %Question 5
    \item
    \begin{enumerate}[leftmargin=!]
        %5a
        \item 
        \begin{proof}
            By the proposition, there exists $C, C_{\infty}\in \RR_{>0}$ such that
            \[||x||\leq C_{\infty} ||x||_{\infty}\leq C_{\infty} C ||x||'\]
            For all $x\in \RR^{n}$. Similarly, there exists $C',C'_{\infty}\in \RR_{>0}$ such that 
            \[||x||' \leq C'_{\infty} ||x||_{\infty} \leq C'_{\infty} C' ||x||\]
            For all $x\in \RR^n$. Therefore, letting $D=C_{\infty}C$ and $D'=  C'_{\infty} C'$ completes the proof.
        \end{proof}
        %5b
        \item 
        \begin{proposition}\label{prop:subaddsqrt}
            Let $a,b\in \RR_{\geq 0}$. Then
            \[\sqrt{a+b} \leq \sqrt{a}+\sqrt{b}.\]
        \end{proposition}
        \begin{proof}[Proof of Proposition~\ref{prop:subaddsqrt}]
            Note that since $a$ and $b$ are non-negative, we have that
            \[a+b \leq a+b+2\sqrt{a}\sqrt{b} = {(\sqrt{a}+\sqrt{b})}^2.\]
            Taking the square root on both sides completes the proof.
        \end{proof}

        Note that by Proposition~\ref{prop:subaddsqrt}, we have the following for any $x\in \RR^2$:
        \[||x||_{2}=\sqrt{x_1^2+x_2^2} \leq \sqrt{x_1^2}+\sqrt{x_2^2} = |x_1|+|x_2| = ||x||_{1}.\]
        Furthermore, by the Cauchy-Schwarz-Bunyakovsky Inequality, we have that
        \[||x||_{1} = |x_1|+|x_2| = 1\cdot |x_1|+1\cdot|x_2| \leq \sqrt{1+1}\sqrt{|x_1|^2+|x_2|^2} = \sqrt{2}||x||_{2}.\]
        Therefore, $C_1=1$ and $C_2 = \sqrt{2}$ gives us the desired result.
    \end{enumerate}

    %Question 6
    \item
    \begin{enumerate}[leftmargin=!]
        %6a
        \item
        For a continuous function f $\in V=C[a,b]$ the $L^1$-norm is
        \[||f||_{1}=\int_{a}^{b}|f(x)|dx\]
        Substituting, we have
        \[||f||_{1}=\int_{0}^{1}|x|dx\]
        Since x is $\geq$ 0 on $[0,1]$,
        \[\int_{0}^{1}|x|dx = \int_{0}^{1}xdx\]
        \[= \left\frac{1}{2}x^{2}\right]_{0}^{1}\]
        \[= \frac{1}{2}\]

        For a continuous function f $\in V=C[a,b]$ the $L^2$-norm is
        \[||f||_{2}=\sqrt{\int_{a}^{b}f(x)^{2}dx}\]
        Substituting, we have
        \[||f||_{2}=\sqrt{\int_{0}^{1}x^{2}dx}\]
        \[=\sqrt{\left\frac{1}{3}x^{3}\right]_{0}^{1}}\]
        \[=\frac{1}{\sqrt{3}}\]

        For a continuous function f $\in V=C[a,b]$ the $L^{\infty}$-norm is
        \[||f||_{\infty}=\max_{a \leq x \leq b}{|f(x)|}\]
        Substituting, we have
        \[||f||_{\infty}=\max_{0 \leq x \leq 1}\{|x|\}\]
        \[=1\]
        
        %6b
        \item
        For continuous functions f, g $\in V=C[a,b]$ the distance between f and g with respect to the $L^1$-norm is
        \[||f-g||_{1}=\int_{a}^{b}|f-g|dx\]
        Substituting, we have
        \[||f-g||_{1}=\int_{0}^{1}|x-(1-x)|dx\]
        \[=\int_{0}^{1}|2x-1|dx\]
        Since 2x-1 is $<$ 0 on $[0,0.5)$ and $geq$ 0 on $[0.5,1]$, we can perform piecewise integration, splitting into $\int_{0}^{0.5}1-2xdx + \int_{0.5}^{1}2x-1dx$
        \[\int_{0}^{0.5}1-2xdx = \leftx-x^{2}\right]_{0}^{0.5}\]
        \[=\frac{1}{4}\]
        \[\int_{0.5}^{1}2x-1dx = \leftx^{2}-x\right]_{0.5}^{1}\]
        \[=(1-1)-(\frac{1}{4}-\frac{1}{2})\]
        \[=\frac{1}{4}\]
        $\frac{1}{4}+\frac{1}{4}=\frac{1}{2}$, thus $||f-g||_{1}=\int_{0}^{1}|2x-1|dx=\frac{1}{2}$ 

        For continuous functions f, g $\in V=C[a,b]$ the distance between f and g with respect to the $L^2$-norm is
        \[||f-g||_{2}=\sqrt{\int_{a}^{b}(f-g)^{2}dx}\]
        Substituting, we have
        \[||f-g||_{2}=\sqrt{\int_{0}^{1}(x-(1-x))^{2}dx}\]
        \[=\sqrt{\int_{0}^{1}(2x-1)^{2}dx}\]
        \[=\sqrt{\int_{0}^{1}4x^{2}-4x+1dx}\]
        \[=\sqrt{\left\frac{4}{3}x^{3}-2x^{2}+x\right]_{0}^{1}}\]
        \[=\sqrt{\frac{4}{3}-2+1}\]
        \[=\sqrt{\frac{1}{3}}=\frac{1}{\sqrt{3}}\]

        For continuous functions f, g $\in V=C[a,b]$ the distance between f and g with respect to the $L^{\infty}$-norm is
        \[||f-g||_{\infty}=\max_{a \leq x \leq b}{|f-g|}\]
        Substituting, we have
        \[||f-g||_{\infty}=\max_{0 \leq x \leq 1}{|x-(1-x)|}\]
        \[=\max_{0 \leq x \leq 1}{|2x-1|}\]
        \[=1\]

    \end{enumerate}
    
\end{enumerate}
\end{document}
