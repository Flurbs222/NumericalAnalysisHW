\documentclass[12pt]{article}
\usepackage{Environments}
\usepackage{Packages}

\title{Numerical Analysis HW8}
\author{Coyne, Dedvukaj, Gao, Karabushin, Lin, Morales}
\date{\today}

\setlength{\headheight}{15pt}

\begin{document}
\pagestyle{fancy}

\fancyhead[L]{Coyne, Dedvukaj, Gao, Karabushin, Lin, Morales}
\fancyhead[R]{Josh Morales}

\begin{center}
\textbf{\Large Homework 8} \\
\text{Due date}: April 3rd, 2025
\end{center}

\begin{enumerate}[leftmargin=0em]
  %Problem 1
  \item
  \begin{enumerate}[leftmargin=!]
    %1a
    \item

    %1b
    \item

    %1c
    \item
  \end{enumerate}

  %Problem 2
  \item
  We are given the data points:
  \((1, 2.31), (2, 2.01), (3, 1.80), (4, 1.66), (5, 1.55), (6, 1.47), (7, 1.41)\)
  We seek a quadratic function of the form:
  \(f(x) = a + bx + cx^2\)
  The normal equations are as such:
  \begin{align}
  \sum y &= a n + b \sum x + c \sum x^2, \\
  \sum xy &= a \sum x + b \sum x^2 + c \sum x^3, \\
  \sum x^2 y &= a \sum x^2 + b \sum x^3 + c \sum x^4.
  \end{align}
  
  The values of $x$, $y$, and the required powers of $x$ are:
  \begin{center}
  \begin{tabular}{c|c|c|c|c|c|c}
  $x$ & $y$ & $x^2$ & $x^3$ & $x^4$ & $xy$ & $x^2y$ \\
  \hline
  1 & 2.31 & 1 & 1 & 1 & 2.31 & 2.31 \\
  2 & 2.01 & 4 & 8 & 16 & 4.02 & 8.04 \\
  3 & 1.80 & 9 & 27 & 81 & 5.40 & 16.20 \\
  4 & 1.66 & 16 & 64 & 256 & 6.64 & 26.56 \\
  5 & 1.55 & 25 & 125 & 625 & 7.75 & 38.75 \\
  6 & 1.47 & 36 & 216 & 1296 & 8.82 & 52.92 \\
  7 & 1.41 & 49 & 343 & 2401 & 9.87 & 69.03 \\
  \hline
  $\sum$ &  & 140 & 784 & 4676 & 44.81 & 213.81 \\
  \end{tabular}
  \end{center}

  The computed sums are:
  \begin{align}
  \sum x &= 28, & \sum x^2 &= 140, & \sum x^3 &= 784, & \sum x^4 &= 4676, \\
  \sum y &= 12.21, & \sum xy &= 44.81, & \sum x^2y &= 213.81.
  \end{align}

  Substituting the computed sums into the normal equations:
  \begin{align}
  12.21 &= 7a + 28b + 140c, \\
  44.81 &= 28a + 140b + 784c, \\
  213.81 &= 140a + 784b + 4676c.
  \end{align}

  We solve using Gaussian elimination:
  \begin{equation}
  \begin{bmatrix}
  7 & 28 & 140 \\
  28 & 140 & 784 \\
  140 & 784 & 4676
  \end{bmatrix}
  \begin{bmatrix} a \\ b \\ c \end{bmatrix} \(=\) 
  \begin{bmatrix} 12.21 \\ 44.81 \\ 213.81 \end{bmatrix}.
  \end{equation}

  Solving this system gives:
  \begin{align}
  a &\approx 2.5929, \\
  b &\approx -0.3258, \\
  c &\approx 0.0227.
  \end{align}

  Thus, the quadratic least squares approximation is:
  \(f(x) = 2.5929 - 0.3258x + 0.0227x^2.\)

  \end{enumerate}
  
  %Problem 3
  \item
  \begin{proof}
    First, note that since $x^2\geq 0$ for all $x\in \RR$ and $f\in C[a,b]$, we have that
    \[\langle f,f \rangle= \int_{a}^{b}f(x)^2\, dx\, \geq 0.\]
    Furthermore, we have that
    \[\langle f, f\rangle = 0 \iff \int_{a}^{b}f(x)^2\, dx\, = 0 \iff f(x)^2 = 0 \iff f(x) = 0\]
    for all $x\in [a,b]$. This is also due to the non-negativity of $f(x)^2$. Next, for any $f,g\in C[a,b]$ we have that
    \[\langle f, g \rangle = \int_{a}^{b} f(x)g(x)\, dx\, = \int_{a}^{b} g(x)f(x)\, dx\, = \langle g, f \rangle.\]

    \noindent Finally, for any $f,g,h \in C[a,b]$ and $c_{1},c_{2}\in \RR$, by the linearity of the Riemann integral, we have that
    \[\langle c_{1}f(x)+c_{2}g(x), h(x)\rangle = \int_{a}^{b} (c_{1}f(x)+c_{2}g(x))h(x)\, dx\, = \int_{a}^{b} (c_{1}f(x)h(x)+c_{2}g(x)h(x))\, dx\,\] 
    \[= c_{1}\int_{a}^{b} f(x)h(x)\, dx\, +c_{2}\int_{a}^{b} g(x)h(x)\, dx\, = c_{1}\langle f,h\rangle+c_{2} \langle g , h\rangle.\]
    Therefore, $\langle \cdot, \cdot \rangle$ is an inner product, which is the desired result. 
  \end{proof}
  
  %Problem 4
  \item
  We are given that $\mathbf{v} = \begin{pmatrix}
    1\\
    2\\
    3\\
    0
  \end{pmatrix}$ and
  $\mathbf{w} = \begin{pmatrix}
    1\\
    -1\\
    2\\
    -2
  \end{pmatrix}$.
  \begin{enumerate}[leftmargin=!]
    %4a
    \item $\langle \mathbf{v} , \mathbf{w} \rangle = 1 * 1 + 3 * (-1) + 2 * 2 + 0 * (-2) = 2$

    %4b
    \item $||\mathbf{v}|| = \sqrt{1^{2} + 3^{2} + 2^{2}} = \sqrt{14}$

    %4c
    \item $\mathbf{v} - \mathbf{w} = \begin{pmatrix}
      0 \\
      4 \\
      0 \\
      2
    \end{pmatrix}$

    $||\mathbf{v} - \mathbf{w}|| = \sqrt{0^{2} + 4^{2} + 0^{2} + 2^{2}} = 2\sqrt{5}$

    %4d
    \item $\cos(\theta) = \frac{\langle \mathbf{v}, \mathbf{w} \rangle}{||\mathbf{v}||\cdot||\mathbf{w}||} = \frac{2}{\sqrt{14} \cdot \sqrt{1^{2} + (-1)^{2} + 2^{2} + (-2)^{2}}} = \frac{2}{\sqrt{140}}$ 
    
    $\theta = \cos^{-1}(\frac{2}{\sqrt{140}}) \approx 1.40095$
  \end{enumerate}


  %Problem 5
  \item
  \begin{enumerate}[leftmargin=!]
    %5a
    \item
    \[ \langle f, g \rangle = \int_{0}^{1}f(x)g(x)dx\]
    \[ = \int_{0}^{1}(x-1)(x+1)dx\]
    \[ = \int_{0}^{1}(x^2-1)dx\]
    \[ = [\frac{x^3}{3} - x]^{1}_0\]
    \[ = (\frac{1}{3} - 1) - (\frac{0^3}{3} - 0) = \frac{-2}{3}\]
    %5b
    \item
    \[ || f - g || = \sqrt{\int_{0}^{1}|(f-g)|dx}\]
    \[ f(x) - g(x) = (x-1) - (x+1) = x - 1 - x - 1 = -2\]
    \[ |f(x) - g(x)| = 2\]
    \[ \int_{0}^{1}2dx = [2x]^{1}_0 = 2 - 0 = 2\]
    %5c
    \item
    The angle between $f$ and $g$ is
    \[ \cos(\theta) = \frac{\langle f, g \rangle}{||f|| \cdot ||g||}\]
    We already know that $\langle f, g \rangle = \frac{-2}{3}$, so we need to find the norm $||f||$.

    \[||f|| = (\int_{0}^{1}(x-1)^2dx)^{\frac{1}{2}} = (\int_{0}^{1}(x^2-2x+1)dx)^{\frac{1}{2}}\]
    \[ = ([\frac{x^3}{3}-2\frac{x^2}{2} + x]^1_0)^\frac{1}{2} = ((\frac{1}{3} - 2\frac{x^2}{2} + x) - (0))^\frac{1}{2}\]
    \[ = (\frac{1}{3} - 1 +1)^{\frac{1}{2}} = \frac{1}{\sqrt{3}}\].

    Now we need to find $||g||$.
    \[ ||g|| = (\int_{0}^{1}(x+1)^2dx)^{\frac{1}{2}} = (\int_{0}^{1}(x^2+2x+1)dx)^{\frac{1}{2}} = ([\frac{x^3}{3} + x^{2} + x]^{1}_0)^{\frac{1}{2}}\]
    \[ = (\frac{1}{3} + 1 + 1)^\frac{1}{2} = \sqrt{\frac{7}{3}}\]

    We can now use both of these norms to find $\theta$:
    \[ \cos(\theta) = \frac{\frac{-2}{3}}{\frac{-1}{3} \cdot \sqrt{\frac{7}{3}}} = \frac{\frac{-2}{3}}{\sqrt{\frac{7}{9}}} = \frac{\frac{-2}{3}}{\frac{\sqrt{7}}{3}} = \frac{-2}{\sqrt{7}}\]

    Now we can solve for $\theta$:
    \[\theta = \cos^{-1}(\frac{-2}{\sqrt{7}}) \approx \cos^{-1}(-0.7559) \approx 139.1^{\circ}\]
    %5d
    \item
    To find a nonzero $h$ that is perpendicular to $f$, we would have to find a function $h(x)$ that satisfies the following:
    \[ \int_{0}^{1}(x-1) \cdot h(x) dx = 0\]
    We can try a simple polynomial $h(x) = ax + b$, and plug it into the above orthogonality condition:
    \[\int_{0}^{1}(x-1)(ax+b)dx = 0\]
    \[(x-1)(ax+b) = (ax^2+bx -ax - b) = ax^2 + (b-a)x - b\]
    \[\int_{0}^{1}(ax^2+(b-a)x-b)dx = a\frac{1}{3}+(b-a)\frac{1}{2}-b = 0\]
    \[2a+3(b-a) - 6b = 0 \rightarrow 2a+3b-3a-6b = 0 \rightarrow -a -3b = 0 \rightarrow a = -3b\]
    Let us pick $b=1$, then $a = -3$, so $h(x) = -3x+1$
  \end{enumerate}

  %Problem 6
  \item
  We know that $\mathbf{v_1} = (1, 2, 2), \mathbf{v_2} = (-1, 0, 2)$, and $\mathbf{v_3} = (0,0,1)$

  \begin{enumerate}[leftmargin=!]
    %6a
    \item
    \[ \mathbf{v_1} = \mathbf{x_1} \text{ and } ||\mathbf{x_1}||_{2} = \sqrt{1^2+2^2+2^2}\]
    \[ ||\mathbf{w}|| = \frac{\mathbf{x_1}}{||\mathbf{x_1}||_2} = (\frac{1}{3}, \frac{2}{3}, \frac{2}{3}) \]
    \[ \mathbf{x_2} = \mathbf{v_2} - \langle \mathbf{v_2}, \mathbf{w_1} \rangle \mathbf{w_1} = (-1, 0, 2) - \langle (-1, 0, 2) , (\frac{1}{3}, \frac{2}{3}, \frac{2}{3}) \rangle\]
    \[ = (-1, 0, 2) - 1(\frac{1}{3}, \frac{2}{3}, \frac{2}{3})\]
    \[ = (\frac{-4}{3}, \frac{-2}{3}, \frac{4}{3})\]   
    \[||\mathbf{x^2}||_2 = \sqrt{(\frac{-4}{3})^{2} + (\frac{-2}{3})^{2} + (\frac{4}{3})^{2}} = 2\] 
    \[ \mathbf{w_2} = \frac{\mathbf{x_2}}{||\mathbf{x_2}||_2} = (\frac{-2}{3}, \frac{-1}{3}, \frac{2}{3})\]
    \[ \mathbf{x_3} = \mathbf{v_3} - \langle \mathbf{v_3}, \mathbf{w_1} \rangle \mathbf{w_1} - \langle \mathbf{v_3}, \mathbf{w_2} \rangle \mathbf{w_2}\]
    \[ \mathbf{x_3} = (0, 0, 1) - \langle (0, 0, 1), (\frac{1}{3}, \frac{2}{3}, \frac{2}{3})\rangle (\frac{1}{3}, \frac{2}{3}, \frac{2}{3}) - \langle (0,0,1) (\frac{-2}{3}, \frac{-1}{3}, \frac{2}{3}) \rangle (\frac{-2}{3}, \frac{-1}{3}, \frac{2}{3})\]
    \[ = (0,0,1) - \frac{2}{3}(\frac{1}{3}, \frac{2}{3}, \frac{2}{3}) - \frac{2}{3}(\frac{-2}{3}, \frac{-1}{3}, \frac{2}{3})\]
    \[ = (\frac{2}{9}, \frac{-2}{9}, \frac{1}{9})\]
    \[ || \mathbf{x_3} ||_2 = \frac{1}{3} \]
    \[ \mathbf{w_3} = \frac{\mathbf{x_3}}{||\mathbf{x_3}||_2} = (\frac{2}{3}, \frac{-2}{3}, \frac{1}{3})\]

    %6b
    \item
    We know $\mathbf{x} = (7, 5, 1)$
    \[ \langle (7, 5, 1), (\frac{1}{3}, \frac{2}{3}, \frac{2}{3}) \rangle = \frac{19}{3}\]
    \[ \langle (7, 5, 1), (\frac{-2}{3}, \frac{-1}{3}, \frac{2}{3}) \rangle = \frac{-17}{3}\]
    \[ \langle (7, 5, 1), (\frac{2}{3}, \frac{-2}{3}, \frac{1}{3}) \rangle = \frac{5}{3}\].

    Now, we can write 
    \[ \frac{19}{3}(\frac{1}{3}, \frac{2}{3}, \frac{2}{3}) - \frac{17}{3}(\frac{-2}{3}, \frac{-1}{3}, \frac{2}{3}) + \frac{5}{3}(\frac{2}{3}, \frac{-2}{3}, \frac{1}{3})\]
  \end{enumerate}
\end{enumerate}
\end{document}
