\documentclass[12pt]{article}
\usepackage{Environments}
\usepackage{Packages}

\title{Numerical Analysis HW8}
\author{Coyne, Dedvukaj, Gao, Karabushin, Lin, Morales}
\date{\today}

\setlength{\headheight}{15pt}

\begin{document}
\pagestyle{fancy}

\fancyhead[L]{Coyne, Dedvukaj, Gao, Karabushin, Lin, Morales}
\fancyhead[R]{Josh Morales}

\begin{center}
\textbf{\Large Homework 8} \\
\text{Due date}: April 3rd, 2025
\end{center}

\begin{enumerate}[leftmargin=0em]
  %Problem 1
  \item
  \begin{enumerate}[leftmargin=!]
    %1a
    \item

    %1b
    \item

    %1c
    \item
  \end{enumerate}

  %Problem 2
  \item
  
  %Problem 3
  \item
  
  %Problem 4
  \item
  We are given that $\mathbf{v} = \begin{pmatrix}
    1\\
    2\\
    3\\
    0
  \end{pmatrix}$ and
  $\mathbf{w} = \begin{pmatrix}
    1\\
    -1\\
    2\\
    -2
  \end{pmatrix}$.
  \begin{enumerate}[leftmargin=!]
    %4a
    \item $\langle \mathbf{v} , \mathbf{w} \rangle = 1 * 1 + 3 * (-1) + 2 * 2 + 0 * (-2) = 2$

    %4b
    \item $||\mathbf{v}|| = \sqrt{1^{2} + 3^{2} + 2^{2}} = \sqrt{14}$

    %4c
    \item $\mathbf{v} - \mathbf{w} = \begin{pmatrix}
      0 \\
      4 \\
      0 \\
      2
    \end{pmatrix}$

    $||\mathbf{v} - \mathbf{w}|| = \sqrt{0^{2} + 4^{2} + 0^{2} + 2^{2}} = 2\sqrt{5}$

    %4d
    \item $\cos(\theta) = \frac{\langle \mathbf{v}, \mathbf{w} \rangle}{||\mathbf{v}||\cdot||\mathbf{w}||} = \frac{2}{\sqrt{14} \cdot \sqrt{1^{2} + (-1)^{2} + 2^{2} + (-2)^{2}}} = \frac{2}{\sqrt{140}}$ 
    
    $\theta = \cos^{-1}(\frac{2}{\sqrt{140}}) \approx 1.40095$
  \end{enumerate}


  %Problem 5
  \item
  \begin{enumerate}[leftmargin=!]
    %5a
    \item

    %5b
    \item

    %5c
    \item

    %5d
    \item
  \end{enumerate}

  %Problem 6
  \item
  We know that $\mathbf{v_1} = (1, 2, 2), \mathbf{v_2} = (-1, 0, 2)$, and $\mathbf{v_3} = (0,0,1)$

  \begin{enumerate}[leftmargin=!]
    %6a
    \item
    \[ \mathbf{v_1} = \mathbf{x_1} \text{ and } ||\mathbf{x_1}||_{2} = \sqrt{1^2+2^2+2^2}\]
    \[ ||\mathbf{w}|| = \frac{\mathbf{x_1}}{||\mathbf{x_1}||_2} = (\frac{1}{3}, \frac{2}{3}, \frac{2}{3}) \]
    \[ \mathbf{x_2} = \mathbf{v_2} - \langle \mathbf{v_2}, \mathbf{w_1} \rangle \mathbf{w_1} = (-1, 0, 2) - \langle (-1, 0, 2) , (\frac{1}{3}, \frac{2}{3}, \frac{2}{3}) \rangle\]
    \[ = (-1, 0, 2) - 1(\frac{1}{3}, \frac{2}{3}, \frac{2}{3})\]
    \[ = (\frac{-4}{3}, \frac{-2}{3}, \frac{4}{3})\]   
    \[||\mathbf{x^2}||_2 = \sqrt{(\frac{-4}{3})^{2} + (\frac{-2}{3})^{2} + (\frac{4}{3})^{2}} = 2\] 
    \[ \mathbf{w_2} = \frac{\mathbf{x_2}}{||\mathbf{x_2}||_2} = (\frac{-2}{3}, \frac{-1}{3}, \frac{2}{3})\]
    \[ \mathbf{x_3} = \mathbf{v_3} - \langle \mathbf{v_3}, \mathbf{w_1} \rangle \mathbf{w_1} - \langle \mathbf{v_3}, \mathbf{w_2} \rangle \mathbf{w_2}\]
    \[ \mathbf{x_3} = (0, 0, 1) - \langle (0, 0, 1), (\frac{1}{3}, \frac{2}{3}, \frac{2}{3})\rangle (\frac{1}{3}, \frac{2}{3}, \frac{2}{3}) - \langle (0,0,1) (\frac{-2}{3}, \frac{-1}{3}, \frac{2}{3}) \rangle (\frac{-2}{3}, \frac{-1}{3}, \frac{2}{3})\]
    \[ = (0,0,1) - \frac{2}{3}(\frac{1}{3}, \frac{2}{3}, \frac{2}{3}) - \frac{2}{3}(\frac{-2}{3}, \frac{-1}{3}, \frac{2}{3})\]
    \[ = (\frac{2}{9}, \frac{-2}{9}, \frac{1}{9})\]
    \[ || \mathbf{x_3} ||_2 = \frac{1}{3} \]
    \[ \mathbf{w_3} = \frac{\mathbf{x_3}}{||\mathbf{x_3}||_2} = (\frac{2}{3}, \frac{-2}{3}, \frac{1}{3})\]

    %6b
    \item
    We know $\mathbf{x} = (7, 5, 1)$
    \[ \langle (7, 5, 1), (\frac{1}{3}, \frac{2}{3}, \frac{2}{3}) \rangle = \frac{19}{3}\]
    \[ \langle (7, 5, 1), (\frac{-2}{3}, \frac{-1}{3}, \frac{2}{3}) \rangle = \frac{-17}{3}\]
    \[ \langle (7, 5, 1), (\frac{2}{3}, \frac{-2}{3}, \frac{1}{3}) \rangle = \frac{5}{3}\].

    Now, we can write 
    \[ \frac{19}{3}(\frac{1}{3}, \frac{2}{3}, \frac{2}{3}) - \frac{17}{3}(\frac{-2}{3}, \frac{-1}{3}, \frac{2}{3}) + \frac{5}{3}(\frac{2}{3}, \frac{-2}{3}, \frac{1}{3})\]
  \end{enumerate}
\end{enumerate}
\end{document}
