\documentclass[12pt]{article}
\usepackage{Environments}
\usepackage{Packages}

\title{Numerical Analysis HW9}
\author{Coyne, Dedvukaj, Gao, Karabushin, Lin, Morales}
\date{\today}

\setlength{\headheight}{15pt}

\begin{document}
\pagestyle{fancy}

\fancyhead[L]{Coyne, Dedvukaj, Gao, Karabushin, Lin, Morales}
\fancyhead[R]{Josh Morales}

\begin{center}
\textbf{\Large Homework 8} \\
\text{Due date}: April 9th, 2025
\end{center}

\begin{enumerate}[leftmargin=0em]
    %Problem 1
    \item 

    %Problem 2
    \item 

    %Problem 3
    \item
    \begin{enumerate}[leftmargin=!]
        %3a
        \item
        %3b
        \item 
    \end{enumerate} 

    %Problem 4
    \item
    \begin{enumerate}[leftmargin=!]
        %4a
        \item Q is orthogonal.
        \[ Q^{t}Q=1\]
        \[\det(Q^tQ) = 1\]
        \[\det(Q^t)\det(Q)=1\]
        Since $\det(Q^t) = \det(Q)$,
        \[det(Q)^2 = 1\]
        \[\det(Q)\pm 1\]
        %4b
        \item
        \[(PQ)^tPQ = Q^tP^tPQ = Q^t(I)Q = Q^tQ=I\].
        So $PQ$ is orthogonal.
    \end{enumerate}

    %Problem 5
    \item
    \begin{enumerate}[leftmargin=!]
        %5a
        \item 
        %5b
        \item 
        %5c
        \item 
    \end{enumerate}

    %Problem 6
    \item We are given the symmetric matrix:

    \[
    A = \begin{bmatrix}
    1 & -1 & 2 \\
    -1 & 1 & 2 \\
    2 & 2 & 2
    \end{bmatrix}
    \]
    
    We want to decompose it as:
    
    \[
    A = Q D Q^T
    \]
    
    To find the eigenvalues, we solve the characteristic polynomial:
    
    \[\det(A - \lambda I) \]
    
    \[ A - \lambda I = \begin{bmatrix}
    1 & -1 & 2 \\
    -1 & 1 & 2 \\
    2 & 2 & 2
    \end{bmatrix}
    -
    \begin{bmatrix}
        \lambda & 0 & 0 \\
        0 & \lambda & 0 \\
        0 & 0 & \lambda
    \end{bmatrix}
    \]
    \[
    = \begin{bmatrix}
        1 - \lambda & -1 & 2 \\
        -1 & 1-\lambda & 2 \\
        2 & 2 & 2-\lambda
    \end{bmatrix}
    \]
    We know can find the determinant of the matrix:
    
    \[(1-\lambda)
    \begin{bmatrix}
    1-\lambda & 2 \\
    2 & 2-\lambda
    \end{bmatrix}
    -(-1)
    \begin{bmatrix}
        -1 & 2 \\
        2 & 2-\lambda
    \end{bmatrix}
    +2 \begin{bmatrix}
        -1 & 1-\lambda \\
        2 & 2
    \end{bmatrix} 
    = 
    -\lambda^{3} + 4\lambda^{2}+4\lambda-16 = -(\lambda - 4)(\lambda - 2)(\lambda + 2)
    \]
    
    The eigenvalues are:
    \[
    \lambda_1 = 4, \quad \lambda_2 = 2, \quad \lambda_3 = -2
    \]
    Solve \( (A - 4I)\vec{v} = 0 \), we get:
    \[
    \vec{v}_1 = \begin{bmatrix} 1 \\ 1 \\ 2 \end{bmatrix}, \quad
    \|\vec{v}_1\| = \sqrt{6}
    \Rightarrow
    q_1 = \frac{1}{\sqrt{6}} \begin{bmatrix} 1 \\ 1 \\ 2 \end{bmatrix}
    \]
    
    Solve \( (A - 2I)\vec{v} = 0 \), we get:
    \[
    \vec{v}_2 = \begin{bmatrix} 1 \\ -1 \\ 0 \end{bmatrix}, \quad
    \|\vec{v}_2\| = \sqrt{2}
    \Rightarrow
    q_2 = \frac{1}{\sqrt{2}} \begin{bmatrix} 1 \\ -1 \\ 0 \end{bmatrix}
    \]
    
    Solve \( (A + 2I)\vec{v} = 0 \), we get:
    \[
    \vec{v}_3 = \begin{bmatrix} 1 \\ 1 \\ -1 \end{bmatrix}, \quad
    \|\vec{v}_3\| = \sqrt{3}
    \Rightarrow
    q_3 = \frac{1}{\sqrt{3}} \begin{bmatrix} 1 \\ 1 \\ -1 \end{bmatrix}
    \]
    
    Orthogonal matrix \( Q \) (columns are the normalized eigenvectors):
    
    \[
    Q =
    \begin{bmatrix}
    \frac{1}{\sqrt{6}} & \frac{1}{\sqrt{2}} & \frac{1}{\sqrt{3}} \\
    \frac{1}{\sqrt{6}} & -\frac{1}{\sqrt{2}} & \frac{1}{\sqrt{3}} \\
    \frac{2}{\sqrt{6}} & 0 & -\frac{1}{\sqrt{3}}
    \end{bmatrix}
    \]
    
    Diagonal matrix \( D \) of eigenvalues:
    
    \[
    D = \begin{bmatrix}
    4 & 0 & 0 \\
    0 & 2 & 0 \\
    0 & 0 & -2
    \end{bmatrix}
    \]
    
\end{enumerate}
\end{document}