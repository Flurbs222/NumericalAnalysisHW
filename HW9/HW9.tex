\documentclass[12pt]{article}
\usepackage{Environments}
\usepackage{Packages}

\title{Numerical Analysis HW9}
\author{Coyne, Dedvukaj, Gao, Karabushin, Lin, Morales}
\date{\today}

\setlength{\headheight}{15pt}

\begin{document}
\pagestyle{fancy}

\fancyhead[L]{Coyne, Dedvukaj, Gao, Karabushin, Lin, Morales}
\fancyhead[R]{Josh Morales}

\begin{center}
\textbf{\Large Homework 8} \\
\text{Due date}: April 9th, 2025
\end{center}

\begin{enumerate}[leftmargin=0em]
    %Problem 1
    \item 

    %Problem 2
    \item 

    %Problem 3
    \item
    \begin{enumerate}[leftmargin=!]
        %3a
        \item
        We seek the continuous Fourier approximation \( S_3(x) \) of the function \( f(x) = e^x \) on the interval \( [-\pi, \pi] \), using the formula:
        \[S_n(x) = \left\langle f(x), 1 \right\rangle \cdot \frac{1}{2} + \sum_{k=1}^{n} \left( \left\langle f(x), \sin(kx) \right\rangle \cdot \sin(kx) + \left\langle f(x), \cos(kx) \right\rangle \cdot \cos(kx) \right)\]
        with the inner product defined as:
        \[\langle f(x), g(x) \rangle = \frac{1}{\pi} \int_{-\pi}^{\pi} f(x) g(x) \, dx\]

        \[a_0 = \langle f(x), 1 \rangle = \frac{1}{\pi} \int_{-\pi}^{\pi} e^x \, dx = \frac{1}{\pi} (e^{\pi} - e^{\pi})\]
        \[\Rightarrow \frac{a_0}{2} = \frac{1}{2\pi}(e^{\pi} - e^{-\pi})\]

        \[a_1 = \frac{1}{\pi} \int_{-\pi}^{\pi} e^x \cos(x) \, dx = \frac{1}{\pi} \left[ \frac{e^x}{2} (\sin(x) +\cos(x)) \right]_{-\pi}^{\pi}\]
        \[= \frac{1}{\pi} \left( -\frac{e^{\pi}}{2} + \frac{e^{-\pi}}{2} \right) = -\frac{1}{2\pi}(e^{\pi} - e^{-\pi})\]

        \[b_1 = \frac{1}{\pi} \int_{-\pi}^{\pi} e^x \sin(x) \, dx = \frac{1}{\pi} \left[ \frac{e^x}{2} (\sin(x) - \cos(x)) \right]_{-\pi}^{\pi}\]
        \[= \frac{1}{\pi} \left( \frac{e^{\pi}}{2} - \frac{e^{-\pi}}{2} \right) = \frac{1}{2\pi}(e^{\pi} - e^{-\pi})\]

        Using:
        \[\int e^x \cos(ax) dx = \frac{e^x (a \sin(ax) + \cos(ax))}{a^2 + 1}, \quad\int e^x \sin(ax) dx = \frac{e^x (\sin(ax) - a \cos(ax))}{a^2 + 1}\]
        We get:
        \[a_2 = \frac{1}{\pi} \left[ \frac{e^x (2 \sin(2x) + \cos(2x))}{5} \right]_{-\pi}^{\pi}= \frac{1}{5\pi}(e^{\pi} - e^{-\pi})\]
        \[b_2 = \frac{1}{\pi} \left[ \frac{e^x (\sin(2x) - 2 \cos(2x))}{5} \right]_{-\pi}^{\pi}= -\frac{2}{5\pi}(e^{\pi} - e^{-\pi})\]

        \[a_3 = \frac{1}{\pi} \left[ \frac{e^x (3 \sin(3x) + \cos(3x))}{10} \right]_{-\pi}^{\pi}= -\frac{1}{10\pi}(e^{\pi} - e^{-\pi})\]
        \[b_3 = \frac{1}{\pi} \left[ \frac{e^x (\sin(3x) - 3 \cos(3x))}{10} \right]_{-\pi}^{\pi}= \frac{3}{10\pi}(e^{\pi} - e^{-\pi})\]

        Evaluating numerically, we have:
        \[S_3(x) \approx 3.6767 - 3.6767\cos(x) + 3.6767\sin(x) + 1.4707\cos(2x) - 2.9414\sin(2x) - 0.7353\cos(3x) + 2.2060\sin(3x)\]

        %3b
        \item
        We must calculate \(\langle y,\phi _{k}\rangle \) and \(\langle y,\psi _{k}\rangle \) for \(k=0,1,2,3\) 
        First, calculate \(\langle y,\phi _{0}\rangle \),
        \[\langle y,\phi _{k}\rangle =\frac{1}{10}\sum _{j=0}^{19}y_{j}\cos (kx_{j})\]
        \(\langle y,\phi _{1}\rangle \), \(\langle y,\phi _{2}\rangle \), and \(\langle y,\phi _{3}\rangle \).
        \[\langle y,\phi _{0}\rangle =\frac{1}{10}\sum _{j=0}^{19}y_{j}\cos (0)=\frac{1}{10}\sum _{j=0}^{19}y_{j}\]
        \[\langle y,\phi _{1}\rangle =\frac{1}{10}\sum _{j=0}^{19}y_{j}\cos (x_{j})\]
        \[\langle y,\phi _{2}\rangle =\frac{1}{10}\sum _{j=0}^{19}y_{j}\cos (2x_{j})\]
        \[\langle y,\phi _{3}\rangle =\frac{1}{10}\sum _{j=0}^{19}y_{j}\cos (3x_{j})\]

        Next, calculate \(\langle y,\psi _{k}\rangle \) for \(k=1,2,3\)
        \[\langle y,\psi _{k}\rangle =\frac{1}{10}\sum _{j=0}^{19}y_{j}\sin (kx_{j})\]
        \[\langle y,\psi _{1}\rangle =\frac{1}{10}\sum _{j=0}^{19}y_{j}\sin (x_{j})\]
        \[\langle y,\psi _{2}\rangle =\frac{1}{10}\sum _{j=0}^{19}y_{j}\sin (2x_{j})\]
        \[\langle y,\psi _{3}\rangle =\frac{1}{10}\sum _{j=0}^{19}y_{j}\sin (3x_{j})\]

        \[S_{3}(x)=\frac{1}{2}\langle y,\phi _{0}\rangle +\sum _{k=1}^{3}(\langle y,\phi _{k}\rangle \cos (kx))+\sum _{k=1}^{3}(\langle y,\psi _{k}\rangle \sin (kx))\]
        \[S_{3}(x)=\frac{1}{2}\langle y,\phi _{0}\rangle +\langle y,\phi _{1}\rangle \cos (x)+\langle y,\phi _{2}\rangle \cos (2x)+\langle y,\phi _{3}\rangle \cos (3x)+\langle y,\psi _{1}\rangle \sin (x)+\langle y,\psi _{2}\rangle \sin (2x)+\langle y,\psi _{3}\rangle \sin (3x)\]
        
        \[S_{3}(x)\approx 3.676+2.963\cos (x)+0.946\cos (2x)+0.164\cos (3x)-1.263\sin (x)-0.740\sin (2x)-0.214\sin (3x)\]
        
    \end{enumerate} 

    %Problem 4
    \item
    \begin{enumerate}[leftmargin=!]
        %4a
        \item Q is orthogonal.
        \[ Q^{t}Q=1\]
        \[\det(Q^tQ) = 1\]
        \[\det(Q^t)\det(Q)=1\]
        Since $\det(Q^t) = \det(Q)$,
        \[\det(Q)^2 = 1\]
        \[\det(Q)\pm 1\]
        %4b
        \item
        \[(PQ)^tPQ = Q^tP^tPQ = Q^t(I)Q = Q^tQ=I\].
        So $PQ$ is orthogonal.
    \end{enumerate}

    %Problem 5
    \item
    \begin{enumerate}[leftmargin=!]
        %5a
        \item 
        %5b
        \item 
        %5c
        \item 
    \end{enumerate}

    %Problem 6
    \item We are given the symmetric matrix:

    \[
    A = \begin{bmatrix}
    1 & -1 & 2 \\
    -1 & 1 & 2 \\
    2 & 2 & 2
    \end{bmatrix}
    \]
    
    We want to decompose it as:
    
    \[
    A = Q D Q^T
    \]
    
    To find the eigenvalues, we solve the characteristic polynomial:
    
    \[\det(A - \lambda I) \]
    
    \[ A - \lambda I = \begin{bmatrix}
    1 & -1 & 2 \\
    -1 & 1 & 2 \\
    2 & 2 & 2
    \end{bmatrix}
    -
    \begin{bmatrix}
        \lambda & 0 & 0 \\
        0 & \lambda & 0 \\
        0 & 0 & \lambda
    \end{bmatrix}
    \]
    \[
    = \begin{bmatrix}
        1 - \lambda & -1 & 2 \\
        -1 & 1-\lambda & 2 \\
        2 & 2 & 2-\lambda
    \end{bmatrix}
    \]
    We know can find the determinant of the matrix:
    
    \[(1-\lambda)
    \begin{bmatrix}
    1-\lambda & 2 \\
    2 & 2-\lambda
    \end{bmatrix}
    -(-1)
    \begin{bmatrix}
        -1 & 2 \\
        2 & 2-\lambda
    \end{bmatrix}
    +2 \begin{bmatrix}
        -1 & 1-\lambda \\
        2 & 2
    \end{bmatrix} 
    = 
    -\lambda^{3} + 4\lambda^{2}+4\lambda-16 = -(\lambda - 4)(\lambda - 2)(\lambda + 2)
    \]
    
    The eigenvalues are:
    \[
    \lambda_1 = 4, \quad \lambda_2 = 2, \quad \lambda_3 = -2
    \]
    Solve \( (A - 4I)\vec{v} = 0 \), we get:
    \[
    \vec{v}_1 = \begin{bmatrix} 1 \\ 1 \\ 2 \end{bmatrix}, \quad
    \|\vec{v}_1\| = \sqrt{6}
    \Rightarrow
    q_1 = \frac{1}{\sqrt{6}} \begin{bmatrix} 1 \\ 1 \\ 2 \end{bmatrix}
    \]
    
    Solve \( (A - 2I)\vec{v} = 0 \), we get:
    \[
    \vec{v}_2 = \begin{bmatrix} 1 \\ -1 \\ 0 \end{bmatrix}, \quad
    \|\vec{v}_2\| = \sqrt{2}
    \Rightarrow
    q_2 = \frac{1}{\sqrt{2}} \begin{bmatrix} 1 \\ -1 \\ 0 \end{bmatrix}
    \]
    
    Solve \( (A + 2I)\vec{v} = 0 \), we get:
    \[
    \vec{v}_3 = \begin{bmatrix} 1 \\ 1 \\ -1 \end{bmatrix}, \quad
    \|\vec{v}_3\| = \sqrt{3}
    \Rightarrow
    q_3 = \frac{1}{\sqrt{3}} \begin{bmatrix} 1 \\ 1 \\ -1 \end{bmatrix}
    \]
    
    Orthogonal matrix \( Q \) (columns are the normalized eigenvectors):
    
    \[
    Q =
    \begin{bmatrix}
    \frac{1}{\sqrt{6}} & \frac{1}{\sqrt{2}} & \frac{1}{\sqrt{3}} \\
    \frac{1}{\sqrt{6}} & -\frac{1}{\sqrt{2}} & \frac{1}{\sqrt{3}} \\
    \frac{2}{\sqrt{6}} & 0 & -\frac{1}{\sqrt{3}}
    \end{bmatrix}
    \]
    
    Diagonal matrix \( D \) of eigenvalues:
    
    \[
    D = \begin{bmatrix}
    4 & 0 & 0 \\
    0 & 2 & 0 \\
    0 & 0 & -2
    \end{bmatrix}
    \]
    
\end{enumerate}
\end{document}
