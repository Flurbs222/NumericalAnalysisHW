\documentclass[12pt]{article}
\usepackage{Environments}
\usepackage{Packages}
\title{Numerical Analysis HW3}
\author{Josh Morales}
\date{\today}
\setlength{\headheight}{15pt}
\begin{document}
\pagestyle{fancy}
\fancyhead[L]{Numerical Analysis HW 3}
\fancyhead[R]{Coyne, Dedvukaj, Gao, Karabushin, Lin, Morales}
\begin{center}
\textbf{\Large Homework 3} \\
\text{Due date}: February 6th, 2025\\
Martin Coyne, Flora Dedvukaj, Jiahao Gao, Anton Karabushin, Zhihan Lin, Joshua Morales
\end{center}
\begin{enumerate}[leftmargin=2em]
    %Question 1
    \item
	\begin{proof}
	  It is sufficient to show that 
	  \[\lim_{n\to \infty}\frac{|p_{n+1}-p|}{|p_{n}-p|}=g'(p)\]
	  since letting $\lambda = g'(p)$ would complete the proof. By the mean value theorem on $[p_n,p]$, or $[p,p_n]$, there exists a $\xi_n\in (p_n,p)$ such that
	   \[
		 \frac{|g(p_n)-g(p)|}{|p_n-p|}=\frac{|p_{n+1}-p|}{|p_{n}-p|}= g'(\xi_n)
	  .\] 
	  Since $p_n \xrightarrow{n\to \infty} p$, we must have	$\xi_{n}\xrightarrow{n\to \infty} p$. Therefore, by the continuity of $g'$, we must have 
	  \[g'(\xi_n)\xrightarrow{n\to \infty} g'(p).\] 
	  Therefore, taking the limit of both sides of the first equation gives
	  \[\lim_{n\to \infty } \frac{|p_{n+1}-p|}{|p_{n}-p|}=\lim_{n\to \infty} g'(\xi_n)=g'(p)\]
	  which is the desired result.
	\end{proof}
    %Question 2
    \item 
    \begin{enumerate}
        %2a
        \item[(a)] Note that
		  \begin{align*}
			&f'(x)=5x^4-7.2x^2+1.28x+1.536 \implies 	f'(0.8)=0\\
			&f''(x) = 20x^3-14.4 x+1.28 \implies f''(0.8) = 0\\
			&f'''(x)=60x^2-14.4 \implies f'''(0.8)=-14.4\neq 0
		  \end{align*}
        Therefore, $0.8$ is a root of multiplicity $3$ of $f$.
        %2b
        \item[(b)]
		  	\begin{center}
		    \begin{tabular}{|c|c|}
		  	\hline
			$n$ & $x_n$\\
			\hline
			$0$ & $2$\\
			\hline
			$1$ & $1.68$\\
			\hline
			$2$ & $1.4363$\\
			\hline
			$3$ & $1.2536$\\
			\hline
			$4$ & $1.1190$\\
			\hline
			$5$ & $1.0216$\\
			\hline
			$6$ & $0.9038$\\
			\hline
			$7$ & $0.9038$\\
			\hline
			$8$ & $0.8703$\\
			\hline
			$9$ & $0.8474$\\
			\hline
			$10$ & $0.8318$\\
			\hline
		    \end{tabular}
        	\end{center}
        %2c
		  \item[(c)] Using $\mu(x)=\frac{f(x)}{f'(x)}$, we see the following
			\begin{center}
			\begin{tabular}{|c|c|}
		  	\hline
			$n$ & $x_n$\\
			\hline
			$0$ & $2$\\
			\hline
			$1$ & $0.6286$\\
			\hline
			$2$ & $0.78835$\\
			\hline
			$3$ & $0.79995$\\
			\hline
			$4$ & $0.80000$\\
			\hline
			$5$ & $0.80000$\\
			\hline
			$6$ & $0.80000$\\
			\hline
			$7$ & $0.80000$\\
			\hline
			$8$ & $0.80000$\\
			\hline
			$9$ & $0.80000$\\
			\hline
			$10$ & $0.80000$\\
			\hline
			\end{tabular}
			\end{center}
			which clearly converges much faster.
    \end{enumerate}


    %Question 3
    \item
    \begin{enumerate}
        %3a
        \item[(a)] 
        
        %3b
        \item[(b)]
        
        %3c
        \item[(c)]
    \end{enumerate}
    
    %Question 4 (Josh)
    \item 
    \begin{proof}
        Suppose there exists, two degree $n$ polynomials, $p_1$ and $p_2$, such that
        \[p_{1}(x_i)=y_i=p_{2}(x_i)\]
        for all $0\leq i\leq n$. It suffices to show that $p_1(x)=p_2(x)$. Therefore, the polynomial, 
        \[f(x)=p_{1}(x)-p_{2}(x)\]
        is a polynomial of degree at most $n$, with $n+1$ distinct roots, $x_0,x_1,\ldots,x_n$. However, by the Fundamental Theorem of Algebra, $f$ must be the $0$ polynomial.\footnote{Since otherwise it would be a non-zero degree $n$ polynomial with more than $n$ distinct roots.}
        Therefore, \[f(x)=0,\] which means that $p_1(x)=p_2(x)$, which is the desired result.
    \end{proof}
    %Question 5
    \item
    \begin{enumerate}
        %5a
        \item[(a)] We can take $(0,7), (2, 11), (3, 28),$ and $(4,63)$ as $(x_0, y_0), (x_1, y_1), ... (x_n, y_n)$ respectively. We find the following:
        \[L_0(x) = \frac{(x-x_{1})(x-x_{2})(x-x_{3})}{(x_{0}-x_{1})(x_{0}-x_{2})(x_{0}-x_{3})}\]
        \[L_1(x) = \frac{(x-x_{0})(x-x_{2})(x-x_{3})}{(x_{1}-x_{0})(x_{1}-x_{2})(x_{1}-x_{3})}\]
        \[L_2(x) = \frac{(x-x_{0})(x-x_{1})(x-x_{3})}{(x_{2}-x_{0})(x_{2}-x_{1})(x_{2}-x_{3})}\]
        \[L_3(x) = \frac{(x-x_{0})(x-x_{1})(x-x_{2})}{(x_{3}-x_{0})(x_{3}-x_{1})(x_{3}-x_{2})}\]
        Plugging in the points from above, we get:
        \[L_0(x) = \frac{(x-2)(x-3)(x-4)}{(-2)(-3)(-4)} = \frac{-1}{24}(x-2)(x-3)(x-4)\]
        \[L_1(x) = \frac{(x-0)(x-3)(x-4)}{(-2)(-1)(-2)} = \frac{1}{4}(x)(x-3)(x-4)\]
        \[L_2(x) = \frac{(x-0)(x-2)(x-4)}{(3)(1)(-1)} = \frac{-1}{3}(x)(x-2)(x-4)\]
        \[L_3(x) = \frac{(x-0)(x-2)(x-3)}{(4)(2)(1)} = \frac{1}{8}(x)(x-2)(x-3)\]
        Using the y-values from above, we get the following interpolation:
        \[P(x)=\frac{-7}{24}(x-2)(x-3)(x-4) 
        + \frac{11}{4}(x)(x-3)(x-4)-\frac{28}{3}(x)(x-2)(x-4)
        +\frac{63}{8}(x)(x-2)(x-3)\]
        Which can be simplified to become 
        \[P(x) = x^3-2x+7\]
        %5b
        \item[(b)]        
        To find the approximation of $f(1)$, we just plug in $1$ for $x$ in $P(x)$:
        \[P(1)=1^2-2(1)+7 = 1 - 2 + 7 = 6\]
        %5c
        \item[(c)]
        \[\int_{0}^{4}x^3-2x+7 = 76\]
    \end{enumerate}

    %Question 6
    \item

    %Question 7
    \item We know that 
   
    \[ |f(x) - p(x)| = \frac{f^{(n+1)}(\xi)}{(n+1)!}\prod_{i=0}^{n}x-x_{i} \]
    
    Since $x_0, x_1, ..., x_n$ are evenly spaced,

    \[ \prod_{i=0}^{n} (x-x_{i}) \leq \frac{1}{4}((\frac{x_{n}-x_{0}}{n})^{n+1}(n!)) \]
    
    So, 
    \[ |f(x)-p(x)| = |(\frac{f^{11}(\xi)}{11!})(\frac{1}{4})(\frac{1.6875-0}{10})^{11}(10!)\]
    $f(x)=\sin(0.16875x)$, so 
    \[ |f(x)-p(x)| = \max(-\frac{((0.16875)^{11}\cos(0.16875)(0))}{11!}, -\frac{((0.16875)^{11}\cos(0.16875)(1.6785))}{11!})  \].

    Since
    \[|-((0.16785)^{11}(\cos(0.16785)(x=0)))| = 3.15996008e^{-9}\]
    and 
    \[|-((0.16785)^{11}(\cos(0.16785)(x=1.6875)))| = 3.15867894e^{-9}\]

    The maximum value of $f^{11}$ is at $x=0$. Then,
    
    \[|(-\frac{(0.16875)^{11}(\cos(0.16875(1.6875)|))}{11!})\frac{1}{4}(0.16875)^{11}(10!) \leq 2.178e^{-19} \].

    Then, the error bound of $|f(x)-p(x)| \leq 2.178(10^{-9})$ on $[0, 1.6875]$.
    
\end{enumerate}
\end{document}
