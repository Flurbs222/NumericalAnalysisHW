\documentclass[12pt]{article}
\usepackage{Environments}
\usepackage{Packages}
\title{Numerical Analysis HW3}
\author{Josh Morales}
\date{\today}
\setlength{\headheight}{15pt}
\begin{document}
\pagestyle{fancy}
\fancyhead[L]{Numerical Analysis HW 2}
\fancyhead[R]{Coyne, Dedvukaj, Gao, Karabushin, Lin, Morales}
\begin{center}
\textbf{\Large Homework 3} \\
\text{Due date}: February 6th, 2025\\
Martin Coyne, Flora Dedvukaj, Jiahao Gao, Anton Karabushin, Zhihan Lin, Joshua Morales
\end{center}
\begin{enumerate}[leftmargin=2em]
    %Question 1
    \item

    %Question 2
    \item 
    \begin{enumerate}
        %2a
        \item[(a)]
        
        %2b
        \item[(b)]
        
        %2c
        \item[(c)]  
    \end{enumerate}


    %Question 3 (Josh)
    \item
    \begin{enumerate}
        %3a
	  \item[(a)] Since $f$ is a ``suitably'' differentiable function, we may assume that $f^{(i)}(x)$ is continuous in some ``suitable'' region around $p$, for all $0\leq i\leq m$. Since $p$ is a root of multiplicity $m$, 
		We must have that $f^{(m-1)}(p)=0$ and $f^{(m)}(p)\neq 0$. Therefore, by repeated application of L'Hopital's Rule $m-1$ times, and by the fact that $f\in C^m$, we have
		\[
		  \lim_{x\to p}\mu(x)=\lim_{x\to p} \frac{f(x)}{f'(x)} = \lim_{x\to p}\frac{f^{(m-1)}(x)}{f^{(m)}(x)}=\frac{f^{(m-1)}(p)}{f^{(m)}(p-1)}	
		.\] 
		which us the desired result.	
		%3b
	  \item[(b)] Note that since $p$ is a root of $f$ with multiplicity $m$, there exists a function\footnote{which is also differentiable in a suitable interval.} $g(x)$ such that 
		  \[f(x)={(x-p)}^m g(x).\] 
		where $g(p)\neq 0$. Therefore, we have 
		  \[
		  \mu(x)=\frac{f(x)}{f'(x)}
		  =\frac{{(x-p)}^m g(x)}{m{(x-p)}^{p-1}g(x)+{(x-p)}^m g'(x)}=\frac{(x-p)g(x)}{mg(x)+(x-p)g'(x)}
	 	 .\] 
		 If we let $h(x):=\frac{g(x)}{mg(x)+(x-p)g'(x)}$, we then have
		 \[h(p)=\frac{g(p)}{mg(p)}=\frac{1}{m}\neq 0.\] 
		Thus, since $\mu(x)=(x-p)h(x)$, and $h(p)\neq 0$, $p$ must be a simple root of $\mu(x)$. Therefore, $\mu'(p)\neq 0$, which is the desired result. 
        %3c
        \item[(c)] For $x\neq p$, by the quotient rule, we have 
		\[\frac{\mu(x)}{\mu'(x)}=\frac{f(x)}{f'(x)}\frac{{f'(x)}^2}{{f'(x)}^2-f(x)f''(x)}=\frac{f(x)f'(x)}{{f'(x)}^2-f(x)f''(x)}\]
		which is the desired result.
    \end{enumerate}
    
    %Question 4 (Josh)
    \item 
    \begin{proof}
        Suppose there exists, two degree $n$ polynomials, $p_1$ and $p_2$, such that
        \[p_{1}(x_i)=y_i=p_{2}(x_i)\]
        for all $0\leq i\leq n$. It suffices to show that $p_1(x)=p_2(x)$. Therefore, the polynomial, 
        \[f(x)=p_{1}(x)-p_{2}(x)\]
        is a polynomial of degree at most $n$, with $n+1$ distinct roots, $x_0,x_1,\ldots,x_n$. However, by the Fundamental Theorem of Algebra, $f$ must be the $0$ polynomial.\footnote{Since otherwise it would be a non-zero degree $n$ polynomial with more than $n$ distinct roots.}
        Therefore, \[f(x)=0,\] which means that $p_1(x)=p_2(x)$, which is the desired result.
    \end{proof}
    %Question 5
    \item
    \begin{enumerate}
        %5a
        \item[(a)] We can take $(0,7), (2, 11), (3, 28),$ and $(4,63)$ as $(x_0, y_0), (x_1, y_1), ... (x_n, y_n)$ respectively. We find the following:
        \[L_0(x) = \frac{(x-x_{1})(x-x_{2})(x-x_{3})}{(x_{0}-x_{1})(x_{0}-x_{2})(x_{0}-x_{3})}\]
        \[L_1(x) = \frac{(x-x_{0})(x-x_{2})(x-x_{3})}{(x_{1}-x_{0})(x_{1}-x_{2})(x_{1}-x_{3})}\]
        \[L_2(x) = \frac{(x-x_{0})(x-x_{1})(x-x_{3})}{(x_{2}-x_{0})(x_{2}-x_{1})(x_{2}-x_{3})}\]
        \[L_3(x) = \frac{(x-x_{0})(x-x_{1})(x-x_{2})}{(x_{3}-x_{0})(x_{3}-x_{1})(x_{3}-x_{2})}\]
        Plugging in the points from above, we get:
        \[L_0(x) = \frac{(x-2)(x-3)(x-4)}{(-2)(-3)(-4)} = \frac{-1}{24}(x-2)(x-3)(x-4)\]
        \[L_1(x) = \frac{(x-0)(x-3)(x-4)}{(-2)(-1)(-2)} = \frac{1}{4}(x)(x-3)(x-4)\]
        \[L_2(x) = \frac{(x-0)(x-2)(x-4)}{(3)(1)(-1)} = \frac{-1}{3}(x)(x-2)(x-4)\]
        \[L_3(x) = \frac{(x-0)(x-2)(x-3)}{(4)(2)(1)} = \frac{1}{8}(x)(x-2)(x-3)\]
        Using the y-values from above, we get the following interpolation:
        \[P(x)=\frac{-7}{24}(x-2)(x-3)(x-4) 
        + \frac{11}{4}(x)(x-3)(x-4)-\frac{28}{3}(x)(x-2)(x-4)
        +\frac{63}{8}(x)(x-2)(x-3)\]
        Which can be simplified to become 
        \[P(x) = x^3-2x+7\]
        %5b
        \item[(b)]        
        To find the approximation of $f(1)$, we just plug in $1$ for $x$ in $P(x)$:
        \[P(1)=1^2-2(1)+7 = 1 - 2 + 7 = 6\]
        %5c
        \item[(c)]
        \[\int_{0}^{4}x^3-2x+7 = 76\]
    \end{enumerate}

    %Question 6
    \item

    %Question 7
    \item
\end{enumerate}
\end{document}
