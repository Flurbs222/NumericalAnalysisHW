\documentclass[12pt]{article}
\usepackage{Environments}
\usepackage{Packages}
\title{Numerical Analysis HW3}
\author{Josh Morales}
\date{\today}
\setlength{\headheight}{15pt}
\begin{document}
\pagestyle{fancy}
\fancyhead[L]{Numerical Analysis HW 2}
\fancyhead[R]{Coyne, Dedvukaj, Gao, Karabushin, Lin, Morales}
\begin{center}
\textbf{\Large Homework 3} \\
\text{Due date}: February 6th, 2025\\
Martin Coyne, Flora Dedvukaj, Jiahao Gao, Anton Karabushin, Zhihan Lin, Joshua Morales
\end{center}
\begin{enumerate}[leftmargin=2em]
    %Question 1
    \item

    %Question 2
    \item 
    \begin{enumerate}
        %2a
        \item[(a)]
        
        %2b
        \item[(b)]
        
        %2c
        \item[(c)]  
    \end{enumerate}


    %Question 3
    \item
    \begin{enumerate}
        %3a
        \item[(a)] 
        
        %3b
        \item[(b)]
        
        %3c
        \item[(c)]
    \end{enumerate}
    
    %Question 4 (Josh)
    \item 
    \begin{proof}
        Suppose there exists, two degree $n$ polynomials, $p_1$ and $p_2$, such that
        \[p_{1}(x_i)=y_i=p_{2}(x_i)\]
        for all $0\leq i\leq n$. It suffices to show that $p_1(x)=p_2(x)$. Therefore, the polynomial, 
        \[f(x)=p_{1}(x)-p_{2}(x)\]
        is a polynomial of degree at most $n$, with $n+1$ distinct roots, $x_0,x_1,\ldots,x_n$. However, by the Fundamental Theorem of Algebra, $f$ must be the $0$ polynomial.\footnote{Since otherwise it would be a non-zero degree $n$ polynomial with more than $n$ distinct roots.}
        Therefore, \[f(x)=0,\] which means that $p_1(x)=p_2(x)$, which is the desired result.
    \end{proof}
    %Question 5
    \item
    \begin{enumerate}
        %5a
        \item[(a)]

        %5b
        \item[(b)]
        
        %5c
        \item[(c)]
    \end{enumerate}

    %Question 6
    \item

    %Question 7
    \item
\end{enumerate}
\end{document}