\documentclass[12pt]{article}
\usepackage{Environments}
\usepackage{Packages}
\title{}
\author{Josh Morales}
\date{\today}
\setlength{\headheight}{15pt}
\begin{document}
\pagestyle{fancy}
\fancyhead[L]{Numerical Analysis HW 5}
\fancyhead[R]{Coyne, Dedvukaj, Gao, Karabushin, Lin, Morales}
\begin{center}
\textbf{\Large Homework } \\
\text{Due date}: \today
\end{center}
\begin{enumerate}[leftmargin=2em]
    %Question 1
    \item
    Let $f (x) = 3xe^x - cos x$. By using the forward-difference formula, the three-point
midpoint formula, and the five-point midpoint formula with $h = 0.1, 0.05,
0.01$, compute approximations of $f'(1.3)$.\\
\\For $x_0=1.3$, the function at the values modified by $h$ are as follows.
\[
\begin{array}{|c|c|c|c|}
\hline
h & \text{Rule} & x & f(x) \\
\hline
0.1  & x_0 - h  & 1.2  & 11.590063167374895  \\
     & x_0 + h  & 1.4  & 16.86187271784739  \\
     & x_0 - 2h & 1.1  & 9.460151757597654   \\
     & x_0 + 2h & 1.5  & 20.096863614853586  \\
\hline
0.05 & x_0 - h  & 1.25 & 12.773463728086636  \\
     & x_0 + h  & 1.35 & 15.403566712229708  \\
     & x_0 - 2h & 1.2  & 11.590063167374895  \\
     & x_0 + 2h & 1.4  & 16.86187271784739  \\
\hline
0.01 & x_0 - h  & 1.29 & 13.781763095706816  \\
     & x_0 + h  & 1.31 & 14.307412656453414  \\
     & x_0 - 2h & 1.28 & 13.524381336554086  \\
     & x_0 + 2h & 1.32 & 14.575773202300644  \\
\hline
\end{array}
\]


\[
f'(x_0) = \frac{1}{h}(f(x_0+h)-f(x_0)) + O(h)
\]
For $h = 0.1$
\[
\hbox{$f'(1.3) = \frac{1}{0.1}(f(1.4)-f(1.3)) = \frac{16.86187271784739-14.042758175090468}{0.1} \approx 28.1911$}
\]
For $h = 0.05$
\[
\hbox{$f'(1.3) = \frac{1}{0.05}(f(1.35)-f(1.3)) = \frac{15.403566712229708-14.042758175090468}{0.05} \approx 27.2162$}
\]
For $h=0.01$
\[
\hbox{$f'(1.3) = \frac{1}{0.01}(f(1.31)-f(1.3)) = \frac{14.307412656453414-14.042758175090468}{0.01} \approx 26.4654$}
\]
\subsection*{Three-Point Midpoint Approximation}
\[
f'(x_0) = \frac{1}{2h}(f(x_0+h)-f(x_0-h)) + O(h^2)
\]
For $h = 0.1$
\[
f'(1.3) \approx \frac{1}{2(0.1)}(16.86187271784739-11.590063167374895) \approx 26.3590
\]
For $h = 0.05$
\[
f'(1.3) \approx \frac{1}{2(0.05)}(15.403566712229708-12.773463728086636) \approx 26.3010
\]
For $h = 0.01$
\[
f'(1.3) \approx \frac{1}{2(0.01)}(14.307412656453414-13.781763095706816) \approx 26.2825
\]

\subsection*{Five-Point Midpoint Approximation}
\[
f'(x_0) = \frac{1}{12h} \big( f(x_0-2h) - 8f(x_0-h) + 8f(x_0+h) - f(x_0+2h) \big) + O(h^4)
\]
For $h = 0.1$
\[
\hbox{$f'(1.3) \approx \frac{(9.460151757597654) - 8(11.590063167374895) + 8(16.86187271784739) - (20.096863614853586)}{12(0.1)}$}
\]
\[
\approx 26.2815
\]
For $h = 0.05$
\[
\hbox{$f'(1.3) \approx \frac{(11.590063167374895) - 8(12.773463728086636) + 8(15.403566712229708) - (16.86187271784739)}{12(0.05)}$}
\]
\[
\approx 26.2817
\]
For $h = 0.01$
\[
\hbox{$f'(1.3) \approx \frac{(13.524381336554086) - 8(13.781763095706816) + 8(14.307412656453414) - (14.575773202300644)}{12(0.01)}$}
\]
\[
\approx 26.2817
\]
\subsection*{Conclusion}
All of these methods improve as $h$ decreases, but those with higher orders converge much faster. The Five-Point Midpoint method is the most precise, but it may be computationally intensive.

    %Question 2
    \item
    Note that using taylor expansions around $x$, we get that
    \[f(x+h)=f(x)+hf'(x)+\frac{h^2}{2}f''(x)+O(h^3)\]
    and
    \[f(x+2h)=f(x)+2hf'(x)+2h^2f''(x)+O(h^3).\]
    Therefore, we have
    \[4f(x+h)-3f(x)-f(x+2h)=\]
    \[ 4\left(f(x)+hf'(x)+\frac{h^2}{2}f''(x)\right)-3f(x)-\left(f(x)+2hf'(x)+2h^2f''(x)\right)+O(h^3)\footnote{If $f=O(h^3)$ and $g=O(h^3)$, then $f+g=O(h^3)$.} = 2hf'(x)+O(h^3).\]
    Dividing by $2h$ gives that the error term equals $O(h^2)$, which is the desired result.

    %Question 3
    \item We can differentiate $y=x^{3}$ as follows:
    \[ f'(x) = \frac{d}{dx}x^{3} = 3x^{2}\].
    Thus the integrand becomes 
    \[\sqrt{1+(3x^{2})^{2}} = \sqrt{1+9x^{4}}\].
    So using Simpson's rule, we need to evaluate 
    \[L = \int_{0}^{1}\sqrt{1+9x^{4}}dx\].
    We can approximate the integral as follows:
    \[\int_{a}^{b}f(x)dx \approx \frac{h}{3}[f(x_0)+4f(x_1)+2f(x_2)+4f(x_3)+2f(x_4)+4f(x_5)+f(x_6)]\]
    where $h = \frac{b-a}{n}$ and $x_i = a+ih$.
    For $n=6$, $a=0$, and $b=1$,
    \[h=\frac{1-0}{6} = \frac{1}{6}\]
    The nodes are
    \[x_0 = 0, \quad x_1 = \frac{1}{6}, \quad x_2 = \frac{2}{6}, \quad x_3 = \frac{3}{6}, \quad x_4 = \frac{4}{6}, \quad x_5 = \frac{5}{6}, \quad x_6 = 1.\]
    So we can evaluate $f(x) = \sqrt{1+9x^4}$ at these points:
    \[f(0) = \sqrt{1 + 9(0)^4} = \sqrt{1} = 1.\]
    \[f\left(\frac{1}{6}\right) = \sqrt{1 + 9\left(\frac{1}{6}\right)^4} = \sqrt{1.00694} \approx 1.00347\]
    \[f\left(\frac{2}{6}\right) = \sqrt{1 + 9\left(\frac{2}{6}\right)^4} = \sqrt{1.05556} \approx 1.02747\]
    \[f\left(\frac{3}{6}\right) = \sqrt{1 + 9\left(\frac{3}{6}\right)^4} = \sqrt{1.5625} = 1.25\]
    \[f\left(\frac{4}{6}\right) = \sqrt{1 + 9\left(\frac{4}{6}\right)^4} = \sqrt{1.7778} \approx 1.3333\]
    \[f\left(\frac{5}{6}\right) = \sqrt{1 + 9\left(\frac{5}{6}\right)^4} = \sqrt{2.3403} \approx 1.53\]
    \[f(1) = \sqrt{1 + 9(1)^4} = \sqrt{10} \approx 3.1623\]
    We can then use the Simpson's rule formula
    \[L \approx \frac{h}{3} \left[ f(0) + 4f\left(\frac{1}{6}\right) + 2f\left(\frac{2}{6}\right) + 4f\left(\frac{3}{6}\right) + 2f\left(\frac{4}{6}\right) + 4f\left(\frac{5}{6}\right) + f(1) \right]\]
    And substitute values to get the following:
    \[L \approx \frac{1}{18} \left[ 1 + 4(1.00347) + 2(1.02747) + 4(1.25) + 2(1.3333) + 4(1.53) + 3.1623 \right]\]
    \[L \approx \frac{1}{18} \left[ 1 + 4.0139 + 2.0549 + 5 + 2.6667 + 6.12 + 3.1623 \right]\]
    \[L \approx \frac{1}{18} \times 24.0178\]
    \[L \approx 1.3343\]


    %Question 4
    \item
    Note that
    \[f^{(4)}(x)= (12-48x^2+16x^4)e^{-x^2}.\]
    On $[0,1]$, the above function attains a maximum at $x=0$, with a value of $12$. Therefore, we have that for some $\xi \in [0,1]$,
    \[|E_n(f)|=\left|\frac{-1}{180}h^4f^{(4)}(\xi)\right|\leq \frac{12}{180}h^4.\]
    Solving for $h$ in the above inequality gives 
    \[h\leq 0.01967\ldots.\]
    Noting that $n=\frac{1}{h}$ then gives
    \[n\geq 50.813\ldots.\]
    Therefore, $n=52$, is the smallest possible $n$ that gives us an error of $\leq 10^{-8}$.
    %Question 5
    \item
    \begin{enumerate}[leftmargin=!]
        %5a
        \item 
        \begin{proof}
            By the extreme value theorem. $f$ has a maximum and minimum on $[a,b]$. Therefore, there exists $x_{\min},x_{\max}\in [a,b]$ such that
            $f(x_{\min})= \min\limits_{a\leq x\leq b} f(x)$ and $f(x\max)= \max\limits_{a\leq x\leq b} f(x)$. Furthermore, by definition of the maximum and minimum, we have
            \[f(x_{\min}) = \frac{nf(x_{\min})}{n} = \frac{\sum\limits_{i=1}^{n} f(x_{\min})}{n} \leq \frac{\sum\limits_{i=1}^{n} f(x_{i})}{n} \leq \frac{\sum\limits_{i=1}^{n} f(x_{\max})}{n}= f(x_{\max}).\]
            Therefore, by the intermediate value theorem, there exists an $c \in [x_{\min},x_{\max}]\subseteq [a,b]$ such that 
            \[f(c)= \frac{\sum\limits_{i=1}^{n} f(x_{i})}{n}\]
            which is the desired result.
        \end{proof}

        %5b
        \item 
        \begin{theorem}[Integral Mean Value Theorem for $g(x)=1$]\label{thm:intMVT} Let $f$ be a continuous function on $[a,b]$. Then there exists a $c\in [a.b]$ such that
            \[f(c)(b-a)=\int_{a}^{b}f(x)dx.\]
        \end{theorem}
        \begin{lemma}\label{lem:nonnegativeint}
            Let $f$ be a Riemann integrable function on $[a,b]$, such that 
            \[f(x)\geq 0\]
            for all $x\in [a,b]$. Then
            \[\int_{a}^{b}f(x)dx \geq 0.\]
        \end{lemma}
        \begin{proof}[Proof of Lemma~\ref{lem:nonnegativeint}]
            Let $P=\{a=x_0,x_1,\ldots,x_n=b\}$ be an arbitrary  partition of $[a,b]$. Since $f(x)\geq 0$ on $[a,b]$, we have that
            \[U(f,P) := \sum_{i=1}^{n} (x_i-x_{i-1})M_i \geq 0\]
            and
            \[L(f,P):= \sum_{i=1}^{n} (x_i-x_{i-1})m_i \geq 0\]
            where $M_i:= \sup\limits_{x\in [x_{i-1},x_i]} f(x)$ and $M_i:= \inf\limits_{x\in [x_{i-1},x_i]} f(x)$. Therefore, since $P$ was arbitrary, we must have that
            \[\overline{\int_{a}^{b}} f(x)dx := \inf\{U(f,P)\, |\, P \text{ is a partition of } [a,b]\} \geq 0.\]
            Since $f$ is Riemann integrable, we finally have that
            \[\int_{a}^{b} f(x)dx = \overline{\int_{a}^{b}} f(x)dx\geq 0\]
            which is the desired result.
        \end{proof}
        \begin{lemma}\label{lem:intcomparison}
            Let $f$ and $g$ be Riemann integrable functions on $[a,b]$ such that for all $x\in [a,b]$,
            \[f(x)\geq g(x).\]
            Then
            \[\int_{a}^{b} f(x)dx\geq \int_{a}^{b} g(x)dx\]
        \end{lemma}
        \begin{proof}[Proof of Lemma~\ref{lem:intcomparison}]
            Consider $h(x):= f(x)-g(x)$. Since $f$ and $g$ are Riemann integrable on $[a,b]$, $h$ is also Riemann integrable on $[a,b]$. Furthermore, since 
            $f(x)\geq g(x)$ for all $x\in [a,b]$,
            \[h(x)\geq 0\]
            for all $x\in [a,b]$. Therefore, by Lemma~\ref{lem:nonnegativeint} and the additivity of the Riemann integral, we have that
            \[\int_{a}^{b}f(x)dx - \int_{a}^{b} g(x)dx \geq 0.\]
            Adding $\int_{a}^{b}g(x)dx$ to both sides completes the proof.
        \end{proof}
        \begin{lemma}\label{lem:corollarytocomparison}
            Let $f$ and $g$ be Riemann integrable functions on $[a,b]$ such that for all $x\in [a,b]$,
            \[f(x)\leq g(x).\]
            Then
            \[\int_{a}^{b} f(x)dx\leq \int_{a}^{b} g(x)dx\]
        \end{lemma}
        \begin{proof}[Proof of Lemma~\ref{lem:corollarytocomparison}]
            Since $f(x)\leq g(x)$ on $[a,b]$, then $-f(x)\geq -g(x)$ on $[a,b]$. Furthermore, since $f$ and $g$ are both Riemann integrable, then $-f$ and $-g$ are also both Riemann integrable. Therefore, by Lemma~\ref{lem:intcomparison} and the linearity of the integral, we have
            \[-\int_{a}^{b}f(x)dx \geq -\int_{a}^{b}g(x)dx.\]
            Multiplying by $-1$ completes the proof.
        \end{proof}
        \newpage
        \begin{lemma}\label{lem:comparisontheorem}
            Let $f$ be a continuous function\footnote{Continuous on $[a,b]$ also implies Riemann integrable on $[a,b]$.} on $[a,b]$. Let\footnote{These values exist by the extreme value theorem.} $m=\min_{a\leq x\leq b} f(x)$ and $M=\max_{a\leq x\leq b}f(x)$. 
            Then
            \[m(b-a)\leq \int_{a}^{b} f(x)dx \leq M(b-a).\]
        \end{lemma}
        \begin{proof}[Proof of Lemma~\ref{lem:comparisontheorem}]
            Since $m\leq f(x)\leq M$ for all $x\in [a,b]$. By Lemma~\ref{lem:intcomparison}, Lemma~\ref{lem:corollarytocomparison}, and properties of the Riemann integral, we have that
            \[m(b-a)=\int_{a}^{b} m dx \leq  \int_{a}^{b} f(x) dx \leq \int_{a}^{b} M dx = M(b-a) \]
            which was the desired result.
        \end{proof}
        \begin{proof}[Proof of Theorem~\ref{thm:intMVT}]
            Since $f$ is continuous on $[a,b]$, $f$ attains a maximum and minimum on $[a,b]$, say $M$ and $m$ respectively. Then by Lemma~\ref{lem:comparisontheorem}, we have that
            \[m(b-a)\leq \int_{a}^{b} f(x) dx\leq M(b-a)\]
            which implies that
            \[m\leq \frac{1}{b-a} \int_{a}^{b} f(x) dx \leq M.\]
            Therefore, by the Intermediate Value Theorem, there exists\footnote{Technically $c\in [x_{\min},x_{\max}]$ where $f(x_{\min})=m$ and $f(x_{\max})= M$. However, we will skip over that detail as $[x_{\min},x_{\max}]\subseteq [a,b]$.} a $c\in [a,b]$ such that
            \[f(c)=\frac{1}{b-a}\int_{a}^{b}f(x)dx.\]
            Multiplying by $(b-a)$ gives the desired result.
        \end{proof}
    \end{enumerate}
    
    %Question 6 (Anton)
    \item 
    \begin{enumerate}[leftmargin=!]
        %6a
        \item We must show that the Simpson Rule $S_2$ has no error for polynomials $1, x, x^{2}, x^{3}$, but has an error for $x^{4}$, over a given interval, say, from $x = -i$ to $x = i$. Let us check for exactness for polynomials of increasing degree.
        We perform the computation directly using $h=\frac{b-a}{2}$:

        \[\int_{a}^{b} 1 dx = b - a\]
        and 
        \[S_2 = \frac{h}{3}\left(1+4+1\right)=2h=2\frac{b-a}{2}=b-a=\int_{a}^{b} 1dx.\]
        Next, we have
        \[\int_{a}^{b} x \, dx\, = \frac{b^2-a^2}{2} = \frac{(b-a)(b+a)}{2}\]
        and
        \[S_2 = \frac{h}{3}\left(a+4\frac{a+b}{2}+b\right)=\frac{h}{3}\left(3a+3b\right)=\frac{(b-a)(b+a)}{2}=\int_{a}^{b}x \, dx\,.\]
        Next, we have
        \[\int_{a}^{b} x^2 \, dx\, = \frac{b^3-a^3}{3}=\frac{(b-a)(b^2+ba+a^2)}{3}\]
        and 
        \[S_2 = \frac{h}{3}\left(a^2+4\frac{{(a+b)}^2}{4}+b^2\right) = \frac{h}{3}(2a^2+2b^2+2ab)= \frac{(b-a)(b^2+ba+a^2)}{3} = \int_{a}^{b} x^2\, dx\,.\]
        Finally, we have
        \[\int_{a}^{b} x^3 \, dx\, = \frac{b^4-a^4}{4} = \frac{(b+a)(b-a)(b^2+a^2)}{4}\]
        and
        \[S_2 = \frac{h}{3}\left(a^3+\frac{{(a+b)}^3}{2}+b^3\right) = \frac{h}{3}\left(\frac{3}{2}a^3+\frac{3}{2}b^3+\frac{3}{2}a^2b+\frac{3}{2}ab^2\right)\]
        \[=\frac{1}{4}(b-a)(b+a)(a^2+b^2)=\int_{a}^{b} x^3 \, dx\,.\]
        Note however, that if we let $a=0$ and $b=1$, and we consider $f(x)=x^4$, we get
        \[\int_{0}^{1} x^4\, dx\, = \frac{1}{5}\]
        but
        \[S_2= \frac{1}{6}\left(0+4\frac{1}{16}+1\right)=\frac{5}{24}\neq \frac{1}{5}.\]
        Therefore, $S_2$ has degree of precision $3$.
    
        %6b
        \item We must show that the error when integrating any polynomial of degree three or less using Simpson's rule is zero, while the error for a polynomial of degree four or higher is non-zero.
        \(|E_{n}(f)| = -\frac{(b-a)}{180}h^{4}f^{(4)}(\xi)\) for some $\xi\in[a,b]$. However, note that for any polynomial of degree less than or equal to $3$, we have $f^{(4)}(x)=0$. Therefore, $E_n(f)=0$. However, $E_n(f)\neq 0$ for $f(x)=x^4$ on $[1,2]$. Therefore, $S_n$ has degree of precision $3$.

        %6c
        \item To find the degree of precision of the approximation formula
        \(\int _{-1}^{1}f(x)dx\approx f(\frac{\sqrt{-3}}{3})+f(\frac{\sqrt{3}}{3})\),
        we must test the approximation formula for polynomials \(f(x)=1,x,x^{2},x^{3},...\) of increasing degree until the formula is no longer exact.
        \(\int _{-1}^{1}1dx=2\)
        \(f(-\frac{\sqrt{3}}{3})+f(\frac{\sqrt{3}}{3})=1+1=2\).
        The formula is exact for \(f(x)=1\).
        \(\int _{-1}^{1}xdx=0\)
        \(f(-\frac{\sqrt{3}}{3})+f(\frac{\sqrt{3}}{3})=-\frac{\sqrt{3}}{3}+\frac{\sqrt{3}}{3}=0\).
        The formula is exact for \(f(x)=x\).
        \(\int _{-1}^{1}x^{2}dx=\frac{x^{3}}{3}\Big|_{-1}^{1}=\frac{1}{3}-(-\frac{1}{3})=\frac{2}{3}\)
        \(f(-\frac{\sqrt{3}}{3})+f(\frac{\sqrt{3}}{3})=(-\frac{\sqrt{3}}{3})^{2}+(\frac{\sqrt{3}}{3})^{2}=\frac{3}{9}+\frac{3}{9}=\frac{2}{3}\).
        The formula is exact for \(f(x)=x^{2}\).
        \(\int _{-1}^{1}x^{3}dx=\frac{x^{4}}{4}\Big|_{-1}^{1}=\frac{1}{4}-\frac{1}{4}=0\)
        \(f(-\frac{\sqrt{3}}{3})+f(\frac{\sqrt{3}}{3})=(-\frac{\sqrt{3}}{3})^{3}+(\frac{\sqrt{3}}{3})^{3}=-\frac{3\sqrt{3}}{27}+\frac{3\sqrt{3}}{27}=0\).
        The formula is exact for \(f(x)=x^{3}\).
        \(\int _{-1}^{1}x^{4}dx=\frac{x^{5}}{5}\Big|_{-1}^{1}=\frac{1}{5}-(-\frac{1}{5})=\frac{2}{5}\)
        \(f(-\frac{\sqrt{3}}{3})+f(\frac{\sqrt{3}}{3})=(-\frac{\sqrt{3}}{3})^{4}+(\frac{\sqrt{3}}{3})^{4}=\frac{9}{81}+\frac{9}{81}=\frac{18}{81}=\frac{2}{9}\)
        The formula is not exact for \(f(x)=x^{4}\) since \(\frac{2}{5}\ne \frac{2}{9}\).
        Thus, the degree of precision for the approximation formula is three.
    \end{enumerate}
\end{enumerate}
\end{document}
