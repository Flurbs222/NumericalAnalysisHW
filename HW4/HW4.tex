\documentclass[12pt]{article}
\usepackage{Environments}
\usepackage{Packages}
\title{Numerical Analysis HW3}
\author{Josh Morales}
\date{\today}
\setlength{\headheight}{15pt}
\begin{document}
\pagestyle{fancy}
\fancyhead[L]{Numerical Analysis HW 3}
\fancyhead[R]{Coyne, Dedvukaj, Gao, Karabushin, Lin, Morales}
\begin{center}
\textbf{\Large Homework 4} \\
\text{Due date}: February 13th, 2025\\
Martin Coyne, Flora Dedvukaj, Anton Karabushin, Zhihan Lin, Joshua Morales
\end{center}
\begin{enumerate}[leftmargin=2em]
    %Question 1
    \item
    \begin{enumerate}
        %1a        
        \item
        We are given the points $(0,7), (2,11), (3,28), (4, 63)$, so $f(x_0) = 7, f(x_1)=11, f(x_2)=28$, $f(x_3)=63$.
        
        The first order divided differences are as follows:
        \[f[x_0, x_1] = \frac{7-11}{0-2} = 2\]
        \[f[x_1, x_2] = \frac{28-11}{3-2} = 17\]
        \[f[x_2, x_3] = \frac{63-28}{4-3} = 35\]
        The second order differences are as follows:
        \[f[x_0, x_1, x_2] = \frac{17-2}{3-0} = 5\]
        \[f[x_1, x_2, x_3] = \frac{35-17}{4-2} = 4\]
        And the third order difference is:
        \[f[x_0, x_1, x_2, x_3] = \frac{9-5}{4-0} = 1\]

        Using the Newton formula,
        \[p_n(x)=f[x_0]+f[x_0,x_1](x-x_0) + f[x_0, x_1, x_2](x-x_0)(x-x_1)+...\]
        We can write $p_1$ as
        \[p_1(x) = 7+2(x-0) = 7+2x\]
        $p_2$ as
        \[p_2(x) = 7+2x+5(x-0)(x-2) = 5x^2-8x+7\]
        and $p_3$ as
        \[p_3(x) = 5x^2-8x+7+1(x-0)(x-2)(x-3) = x^3-2x+7\]

        %1b
        \item 
        \[ \int_{0}^{4}(x^3-2x+7)dx \]
        \[ = \int_{0}^{4}x^3dx - \int_{0}^{4}2xdx + \int_{0}^{4}7dx\]
        \[ = (\frac{x^4}{4} - x^2 + 7)\vert_{0}^{4}\]
        \[ = (64 - 16 + 28) - 0 = 76\]
    \end{enumerate}

    %Question 2
    \item
    \begin{enumerate}
        \item We know that the formula for the $k^{th}$ divided difference is as follows:
        \[f[x_{0},x_{1}, ..., x_{k}] = \frac{f[x_{1}, x_{2},...,x_{k}]-f[x_{0}, x_{1},...,x_{k-1}]}{x_{k}-x_{0}}\]
        Given the table, we can calculate the first divided differences as
        \[f[x_{0}, x_{1}] = \frac{f(x_{1})-f(x_0)}{x_1-x_0} = \frac{4-1}{1-(-2)} = \frac{3}{1} = 3\] 
        \[f[x_{1}, x_{2}] = \frac{f(x_{2}) - f(x_{1})}{x_{2}-x_{1}} = \frac{11-4}{0-(-1)} = \frac{7}{1} = 7\]
        \[f[x_{2}, x_{3}] = \frac{f(x_{3})-f(x_{2})}{x_{3}-x_{2}} = \frac{16-11}{1-0} = \frac{5}{1} = 5\]
        \[f[x_{3}, x_{4}] = \frac{f(x_{4})-f(x_{3})}{x_{4}-x_{3}} = \frac{13-16}{2-1} = \frac{-3}{1} = -3\]
        \[f[x_{4}, x_{5}] = \frac{f(x_{5})-f(x_{4})}{x_{5}-x_{4}} = \frac{-4-13}{3-2} = \frac{-17}{1} = -17\]
        The second divided differences are:
        \[f[x_{0}, x_{1}, x_{2}] = \frac{f[x_{1}, x_{2}] - f[x_{0}, x_{1}]}{x_{2}-x_{0}} = \frac{7-3}{0-(-2)} = \frac{4}{2} = 2\]
        \[f[x_{1}, x_{2}, x_{3}] = \frac{f[x_{2}, x_{3}] - f[x_{1}, x_{2}]}{x_{3}-x_{1}} = \frac{5-7}{1-(-1)} = \frac{-2}{2} = -1\]
        \[f[x_{2}. x_{3}, x_{4}] = \frac{f[x_{3}, x_{4}] - f[x_{2}, x_{3}]}{x_{4}-x_{2}} = \frac{-3-5}{2-0} = \frac{-8}{2} = -4\]
        \[f[x_{3}, x_{4}, x_{5}] = \frac{f[x_{4}, x_{5}] - f[x_{3}, x_{4}]}{x_{5} - x_{3}} = \frac{-17-(-3)}{3-1} = \frac{-14}{2} = -7\]
        The third divided differences are
        \[f[x_{0}, x_{1}, x_{2}, x_{3}] = \frac{f[x_{1}, x_{2}, x_{3}] - f[x_{0}, x_{1}, x_{2}]}{x_{3}-x_{0}} = \frac{1-2}{1-(-2)} = \frac{-3}{3} = -1\]
        \[f[x_{1}, x_{2}, x_{3}, x_{4}] = \frac{f[x_{2}, x_{3}, x_{4}] - f[x_{1}, x_{2}, x_{3}]}{x_{4}-x_{1}} = \frac{-4-(-1)}{2-(-1)} = \frac{-3}{3} = -1\]
        \[f[x_{2}, x_{3}, x_{4}, x_{5}] = \frac{f[x_{3}, x_{4}, x_{5}] - f[x_{2}, x_{3}, x_{4}]}{x_{5}-x_{2}} = \frac{-7-(-4)}{3-0} = \frac{-3}{3} = -1\]
        The fourth divided differences are:
        \[f[x_{0}, x_{1}, x_{2}, x_{3}, x_{4}] = \frac{f[x_{1}, x_{2}, x_{3}, x_{4}] - f[x_{0}, x_{1}, x_{2}, x_{3}]}{x_{4}-x_{0}} = \frac{-1-(-1)}{2-(-2)} = \frac{0}{4} = 0\]
        \[f[x_{1}, x_{2}, x_{3}, x_{4}, x_{5}] = \frac{f[x_{2}, x_{3}, x_{4}, x_{5}] - f[x_{1}, x_{2}, x_{3}, x_{4}]}{x_{5}-x_{1}} = \frac{-1-(-1)}{3-(-1)} = \frac{0}{4} = 0\]

        %2b
        \item We will construct the Newton Interpolation formula
        \[ P(x) = f[x_{0}] + \sum_{k=1}^{n}f[x_{0}, x_{1}, ..., x_{k}](x-x_{1})(x-x_{2})... (x-x_{k-1})\]
        using the divided differences calculated in (a).
        \[P(x) = 1 + 3(x+2) + 2(x+2)(x+1) - 1(x+2)(x+1)x\]
        \[P(x) = 1 + 3x + 6 + 2x^{2}+6x + 4 - x^3 - 3x^2 - 2x\]
        \[P(x) = -x^3+(2x^2-3x^2)+ (3x+6x-2x)+(1+6+4)\]
        \[P(x) = -x^3-x^2+7x+11\].
    \end{enumerate}

    %Question 3
    \item
Find the natural cubic spline S(x) that inter-
polates:
\((1,3)\), \((2,1)\), \((3,2)\), \((4,3)\), \((5,2)\) 

\subsection*{Compute \( h_j \) (Step Sizes)}

The step size between each consecutive \( x \)-value is:
\[
h_j = x_{j+1} - x_j
\]

Since the points are equally spaced,
\[
h_1 = h_2 = h_3 = h_4 = 1.
\]

\subsection*{Compute \( \alpha_j \)}

For each \( j = 1, 2, 3 \), we compute:
\[
\alpha_j = \frac{3}{h_j}(a_{j+1} - a_j) - \frac{3}{h_{j-1}}(a_j - a_{j-1})
\]

Substituting the function values \( a_j = f(x_j) \):
\[
\alpha_1 = \frac{3}{1}(2 - 1) - \frac{3}{1}(1 - 3) = 3 + 6 = 9.
\]
\[
\alpha_2 = \frac{3}{1}(3 - 2) - \frac{3}{1}(2 - 1) = 3 - 3 = 0.
\]
\[
\alpha_3 = \frac{3}{1}(2 - 3) - \frac{3}{1}(3 - 2) = -3 - 3 = -6.
\]

\subsection*{Solve the Tridiagonal System for \( c_j \)}

The system is:
\[
\begin{bmatrix}
2 & 1 & 0 & 0 \\
1 & 4 & 1 & 0 \\
0 & 1 & 4 & 1 \\
0 & 0 & 1 & 2
\end{bmatrix}
\begin{bmatrix}
c_1 \\
c_2 \\
c_3 \\
c_4
\end{bmatrix}
=
\begin{bmatrix}
9 \\
0 \\
-6 \\
0
\end{bmatrix}.
\]

Solving this tridiagonal system (using Gaussian elimination), we get:
\[
c_1 = 3, \quad c_2 = 1, \quad c_3 = -1, \quad c_4 = -3.
\]

\subsection*{Compute \( b_j \) and \( d_j \)}

\[
b_j = \frac{a_{j+1} - a_j}{h_j} - \frac{h_j}{3}(c_{j+1} + 2c_j).
\]
\[
d_j = \frac{c_{j+1} - c_j}{3h_j}.
\]

Substituting values:
\[
b_1 = \frac{1 - 3}{1} - \frac{1}{3}(1 + 2(3)) = -2 - \frac{7}{3} = -\frac{13}{3}.
\]
\[
b_2 = \frac{2 - 1}{1} - \frac{1}{3}(-1 + 2(1)) = 1 - \frac{1}{3} = \frac{2}{3}.
\]
\[
b_3 = \frac{3 - 2}{1} - \frac{1}{3}(-3 + 2(-1)) = 1 - \left(-\frac{5}{3}\right) = \frac{8}{3}.
\]
\[
b_4 = \frac{2 - 3}{1} - \frac{1}{3}(0 + 2(-3)) = -1 + 2 = 1.
\]

And for \( d_j \):
\[
d_1 = \frac{1 - 3}{3(1)} = -\frac{2}{3}.
\]
\[
d_2 = \frac{-1 - 1}{3(1)} = -\frac{2}{3}.
\]
\[
d_3 = \frac{-3 - (-1)}{3(1)} = -\frac{2}{3}.
\]
\[
d_4 = \frac{0 - (-3)}{3(1)} = \frac{3}{3} = 1.
\]

\subsection*{Final Spline Functions}

\[
S_1(x) = 3 - \frac{13}{3}(x - 1) + 3(x - 1)^2 - \frac{2}{3}(x - 1)^3, \quad 1 \leq x \leq 2.
\]
\[
S_2(x) = 1 + \frac{2}{3}(x - 2) + 1(x - 2)^2 - \frac{2}{3}(x - 2)^3, \quad 2 \leq x \leq 3.
\]
\[
S_3(x) = 2 + \frac{8}{3}(x - 3) - 1(x - 3)^2 - \frac{2}{3}(x - 3)^3, \quad 3 \leq x \leq 4.
\]
\[
S_4(x) = 3 + 1(x - 4) - 3(x - 4)^2 + 1(x - 4)^3, \quad 4 \leq x \leq 5.
\]

    %Question 4
    \item
    \begin{enumerate}
        %4a
        \item
        We are given the following data points:
        \[ (x_0, y_0) = (0.1, -0.29004996), \quad (x_1, y_1) = (0.2, -0.56079734), \quad (x_2, y_2) = (0.3, -0.81401972).   \]
        The step sizes are:
        \[h_1 = x_1 - x_0 = 0.2 - 0.1 = 0.1, \quad h_2 = x_2 - x_1 = 0.3 - 0.2 = 0.1.\]
        Using the cubic spline formulation, we compute the right-hand side: \[d_1 = \frac{6}{h_1} \left( \frac{y_2 - y_1}{h_2} - \frac{y_1 - y_0}{h_1} \right).\]
        \[= \frac{6}{0.1} \left( \frac{-0.81401972 + 0.56079734}{0.1} - \frac{-0.56079734 + 0.29004996}{0.1} \right).\]
        \[= 60 \left( \frac{-0.25322238}{0.1} - \frac{-0.27074738}{0.1} \right).\]
        \[= 60 (-2.5322238 + 2.7074738) = 60 (0.17525) = 10.515.\]
        Since this is a natural spline the boundary conditions are:
        \[M_0 = 0, \quad M_2 = 0.\]
        Thus, solving for \( M_1 \):
        \[2(h_1 + h_2) M_1 = d_1.\]
        \[2(0.1 + 0.1) M_1 = 10.515.\]
        \[0.4 M_1 = 10.515.\]
        \[M_1 = 26.2875.\]
        The cubic spline equation takes the form:
        \[S_i(x) = a_i + b_i (x - x_i) + c_i (x - x_i)^2 + d_i (x - x_i)^3.\]
        For \( S_1(x) \), in \( [0.1, 0.2] \)
        \[a_1 = y_0 = -0.29004996.\]
        \[b_1 = \frac{y_{1} - y_0}{h_1} - \frac{h_1}{6} (M_1 + 2M_0).\]
        \[= \frac{-0.56079734 + 0.29004996}{0.1} - \frac{0.1}{6} (26.2875).\]
        \[= -2.7074738 - 0.438125 = -3.1456.\]
        \[c_1 = \frac{M_0}{2} = 0.\]
        \[d_1 = \frac{M_1 - M_0}{6h_1} = \frac{26.2875 - 0}{6(0.1)} = \frac{26.2875}{0.6} = 4.3813.\]
        Thus, the first spline equation is:
        \[S_1(x) = 4.3813 (x - 0.1)^3 - 1.3144 (x - 0.1)^2 - 2.6198 (x - 0.1) - 1.9303 \times 10^{-2}.\]
        For \( S_2(x) \), in \( (0.2, 0.3] \)
        \[a_2 = y_1 = -0.56079734.\]
        \[b_2 = \frac{y_2 - y_1}{h_2} - \frac{h_2}{6} (M_2 + 2M_1).\]
        \[= \frac{-0.81401972 + 0.56079734}{0.1} - \frac{0.1}{6} (0 + 2(26.2875)).\]
        \[= -2.5322238 - 0.87625 = -3.4085.\]
        \[c_2 = \frac{M_1}{2} = \frac{26.2875}{2} = 13.1438.\]
        \[d_2 = \frac{M_2 - M_1}{6h_2} = \frac{0 - 26.2875}{6(0.1)} = \frac{-26.2875}{0.6} = -4.3813.\]
        Thus, the second spline equation is:
        \[S_2(x) = -4.3813 (x - 0.2)^3 + 3.9431 (x - 0.2)^2 - 3.6713 (x - 0.2) + 5.0797 \times 10^{-2}.\]
        So \[
            S(x) =
            \begin{cases} 
            4.3813 \cdot x^3 - 1.3144 \cdot x^2 - 2.6198 \cdot x - 1.9303 \times 10^{-2}, & \text{if } x \in [0.1,0.2], \\
            -4.3813 \cdot x^3 + 3.9431 \cdot x^2 - 3.6713 \cdot x + 5.0797 \times 10^{-2}, & \text{if } x \in [0.2,0.3].
            \end{cases}
            \]
        %4b
        \item
        Since we must find $f(0.18)$ and $0.18 \in [0.1,0.2]$, we use $S_2(x)$.
        \[S(0.18) = 4.3813(0.18)^3-1.3144(0.18)^2-2.6198(0.18)-1.9303 \cdot 10^{-2}\]
        \[S(0.18) = -0.507873\]
        \[S'(x)=\frac{d}{dx}(4.3813x^3-1.3144x^2-2.6198x-0.019303)\]
        \[=13.1439x^2-2.6288x-2.6198\]
        Substituting $x=0.18$:
        \[S'_1(0.18) = 13.1439(0.18)^2-2.6288(0.18)-2.6198 = -2.6671\]
        To find the error
        \[f(0.18) = (0.18)^2 \cos(0.18) - 3(0.18).\]
        \[= 0.0324 \cos(0.18) - 0.54.\]
        \[= 0.0324 \times 0.983 - 0.54.\]
        \[= 0.03186 - 0.54 = -0.50814.\]
        This means that the error is
        \[| f(0.18) - S(0.18) | = |-0.50814 - (-0.507873)|.\]
        \[= | -0.000267 | = 0.000267.\]
        For $S'(x)$, we compute
        \[f'(x) = 2x \cos x - x^2 \sin x - 3.\]
        \[f'(0.18) = 2(0.18) \cos(0.18) - (0.18)^2 \sin(0.18) - 3.\]
        \[= 0.36 \times 0.983 - 0.0324 \times 0.179 - 3.\]
        \[= 0.35388 - 0.0057996 - 3.\]
        \[= -2.6519.\]
        Therefore the error is
        \[| f'(0.18) - S'(0.18) | = |-2.6519 - (-2.6671)|.\]
        \[= | 0.0152 |.\]

    \end{enumerate}

    %Question 5 (Anton)
    \item
    \begin{enumerate}
        A natural cubic spline has second derivatives equal to \(0\) at the endpoints. The second derivative of a cubic 
        polynomial is a linear function. Let \(f(x)=ax^{3}+bx^{2}+cx+d\) be a general cubic polynomial, where \(a\ne 0\). 
        The first derivative is \(f^{\prime }(x)=3ax^{2}+2bx+c\). The second derivative is \(f^{\prime \prime }
        (x)=6ax+2b\). For \(f(x)\) to be its own natural cubic spline, \(f^{\prime \prime }(x_{0})=0\) and 
        \(f^{\prime\prime }(x_{1})=0\). This means \(6ax_{0}+2b=0\) and \(6ax_{1}+2b=0\). Subtracting the two equations, we 
        get \(6a(x_{1}-x_{0})=0\). Since \(x_{1}\ne x_{0}\), it must be that \(a=0\). If \(a=0\), then \(2b=0\), so 
        \(b=0\). This contradicts the assumption that \(f(x)\) is a cubic polynomial (\(a\ne 0\)). Therefore, \(f(x)\) 
        cannot be its own natural cubic spline unless \(f^{\prime \prime }(x)\) is identically zero, which means \(f(x)\) 
        is at most a linear function.
    \end{enumerate}
    
    %Question 6 (Joshua)
    \item
    \begin{enumerate}
        %6a
        \item
        \begin{proof}
            Note that by the triangle inequality, we have
            \[|m_{kk}x_k|=\left| \sum_{j\neq k} m_{kj}x_{j}\right|\leq \sum_{j\neq k} |m_{kj}||x_j|.\]
            By definition, $k$ was chosen such that $|x_{k}|= \max\limits_{1\leq j\leq n} |x_j|$. Therefore, since $|m_{kj}|$ is non-negative for all $j\neq k$, we must have
            \[|m_{kk}||x_k|=|m_{kk}x_k|\leq \sum_{j\neq k} |m_{kj}||x_j| \leq \sum_{j\neq k} |m_{kj}||x_k|=|x_k|\sum_{j\neq k} |m_{kj}|.\]
            Dividing by $|x_k|$ gives
            \[|m_{kk}|\leq \sum_{j\neq k} |m_{kj}|\]
            which is the desired result.
        \end{proof}
        %6b
        \item
        We may first assume that $h_i>0$ for all $i$, since we may simply choose the nodes $x_i$ in increasing order\footnote{Also note that we assume each node is distinct, therefore, $h_{i}\neq 0$ for all i}.
        Note that the matrix given by
        \[A_{ij}=\begin{cases}
            2(h_i+h_{i+1}) & \text{ if } i=j\\
            h_i & \text{ if } j = i-1 \text{ or } j= i+1\\
            0 & \text{otherwise}
        \end{cases}\]

        satisfies the following for each $i$:
        \[|A_{ii}|=|2(h_{i}+h_{i+1})|=2(h_{i}+h_{i+1})>2h_{i}=\sum_{j\neq i} A_{ij}.\]

        Therefore, $A$ is strictly diagonally dominant, and thus invertible. 

        \bigskip

        Furthermore, note that since $a_j=f(x_{j})$, the constant coefficients, $a_j$ are uniquely determined\footnote{That is to say, for the family $\{S_j\}_{0\leq j \leq n-1}$ to be a cubic spline on $f$, the conditions derived in class must be satisfied. Showing that the coefficients which satisfy these conditions are unique, then shows that the cubic spline is unique. This will be assumed from now on.} by $f$. Since $A$ is invertible, there is a unique solution to the system
        \[A\mathbf{x}=\mathbf{b}\]
        where
        \begin{align*}
            \mathbf{x} = \begin{pmatrix}
                c_1 \\ 
                c_2 \\ 
                \vdots \\
                c_{n-1}\\
            \end{pmatrix} & & \text{and} && \mathbf{b}=
            \begin{pmatrix}
                \frac{3}{h_{1}}(a_{2}-a_{1}) - \frac{3}{h_{0}}(a_{1}-a_{0})\\
                \frac{3}{h_{2}}(a_{3}-a_{2}) - \frac{3}{h_{1}}(a_{2}-a_{1})\\ 
                \vdots\\
                \frac{3}{h_{n-1}}(a_{n}-a_{n-1})- \frac{3}{h_{n-2}}(a_{n-1}-a_{n-2})
            \end{pmatrix}.
        \end{align*}
        Therefore, since $a_j$ is uniquely determined by $f$, so is\footnote{We also note that $c_0=c_n=0$, so they are also unique.} $c_j$, and by extension 
        \[b_j:= \frac{1}{h_{j}}(a_{j+1}-a_{j})-\frac{h_{j}}{3}(c_{j+1}+2c_{})\]
        and 
        \[d_j:=\frac{1}{3h_{j}}(c_{j+1}-c_{j})\]
        are also uniquely determined by $f$. Therefore, the family $\{S_{j}\}_{0\leq j\leq n-1}$ is the unique cubic spline on $f$, which is the desired result.
    \end{enumerate}
\end{enumerate}
\end{document}
