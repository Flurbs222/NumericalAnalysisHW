\documentclass[12pt]{article}
\usepackage{Environments}
\usepackage{Packages}
\title{Numerical Analysis HW3}
\author{Josh Morales}
\date{\today}
\setlength{\headheight}{15pt}
\begin{document}
\pagestyle{fancy}
\fancyhead[L]{Numerical Analysis HW 3}
\fancyhead[R]{Coyne, Dedvukaj, Gao, Karabushin, Lin, Morales}
\begin{center}
\textbf{\Large Homework 4} \\
\text{Due date}: February 13th, 2025\\
Martin Coyne, Flora Dedvukaj, Jiahao Gao, Anton Karabushin, Zhihan Lin, Joshua Morales
\end{center}
\begin{enumerate}[leftmargin=2em]
    %Question 1
    \item
    \begin{enumerate}
        %1a        
        \item
        We are given the points $(0,7), (2,11), (3,28), (4, 63)$, so $f(x_0) = 7, f(x_1)=11, f(x_2)=28$, $f(x_3)=63$.
        
        The first order divided differences are as follows:
        \[f[x_0, x_1] = \frac{7-11}{0-2} = 2\]
        \[f[x_1, x_2] = \frac{28-11}{3-2} = 17\]
        \[f[x_2, x_3] = \frac{63-28}{4-3} = 35\]
        The second order differences are as follows:
        \[f[x_0, x_1, x_2] = \frac{17-2}{3-0} = 5\]
        \[f[x_1, x_2, x_3] = \frac{35-17}{4-2} = 4\]
        And the third order difference is:
        \[f[x_0, x_1, x_2, x_3] = \frac{9-5}{4-0} = 1\]

        Using the Newton formula,
        \[p_n(x)=f[x_0]+f[x_0,x_1](x-x_0) + f[x_0, x_1, x_2](x-x_0)(x-x_1)+...\]
        We can write $p_1$ as
        \[p_1(x) = 7+2(x-0) = 7+2x\]
        $p_2$ as
        \[p_2(x) = 7+2x+5(x-0)(x-2) = 5x^2-8x+7\]
        and $p_3$ as
        \[p_3(x) = 5x^2-8x+7+1(x-0)(x-2)(x-3) = x^3-2x+7\]

        %1b
        \item 
        \[ \int_{0}^{4}(x^3-2x+7)dx \]
        \[ = \int_{0}^{4}x^3dx - \int_{0}^{4}2xdx + \int_{0}^{4}7dx\]
        \[ = (\frac{x^4}{4} - x^2 + 7)\vert_{0}^{4}\]
        \[ = (64 - 16 + 28) - 0 = 76\]
    \end{enumerate}

    %Question 2
    \item
    \begin{enumerate}
        \item We know that the formula for the $k^{th}$ divided difference is as follows:
        \[f[x_{0},x_{1}, ..., x_{k}] = \frac{f[x_{1}, x_{2},...,x_{k}]-f[x_{0}, x_{1},...,x_{k-1}]}{x_{k}-x_{0}}\]
        Given the table, we can calculate the first divided differences as
        \[f[x_{0}, x_{1}] = \frac{f(x_{1})-f(x_0)}{x_1-x_0} = \frac{4-1}{1-(-2)} = \frac{3}{1} = 3\] 
        \[f[x_{1}, x_{2}] = \frac{f(x_{2}) - f(x_{1})}{x_{2}-x_{1}} = \frac{11-4}{0-(-1)} = \frac{7}{1} = 7\]
        \[f[x_{2}, x_{3}] = \frac{f(x_{3})-f(x_{2})}{x_{3}-x_{2}} = \frac{16-11}{1-0} = \frac{5}{1} = 5\]
        \[f[x_{3}, x_{4}] = \frac{f(x_{4})-f(x_{3})}{x_{4}-x_{3}} = \frac{13-16}{2-1} = \frac{-3}{1} = -3\]
        \[f[x_{4}, x_{5}] = \frac{f(x_{5})-f(x_{4})}{x_{5}-x_{4}} = \frac{-4-13}{3-2} = \frac{-17}{1} = -17\]
        The second divided differences are:
        \[f[x_{0}, x_{1}, x_{2}] = \frac{f[x_{1}, x_{2}] - f[x_{0}, x_{1}]}{x_{2}-x_{0}} = \frac{7-3}{0-(-2)} = \frac{4}{2} = 2\]
        \[f[x_{1}, x_{2}, x_{3}] = \frac{f[x_{2}, x_{3}] - f[x_{1}, x_{2}]}{x_{3}-x_{1}} = \frac{5-7}{1-(-1)} = \frac{-2}{2} = -1\]
        \[f[x_{2}. x_{3}, x_{4}] = \frac{f[x_{3}, x_{4}] - f[x_{2}, x_{3}]}{x_{4}-x_{2}} = \frac{-3-5}{2-0} = \frac{-8}{2} = -4\]
        \[f[x_{3}, x_{4}, x_{5}] = \frac{f[x_{4}, x_{5}] - f[x_{3}, x_{4}]}{x_{5} - x_{3}} = \frac{-17-(-3)}{3-1} = \frac{-14}{2} = -7\]
        The third divided differences are
        \[f[x_{0}, x_{1}, x_{2}, x_{3}] = \frac{f[x_{1}, x_{2}, x_{3}] - f[x_{0}, x_{1}, x_{2}]}{x_{3}-x_{0}} = \frac{1-2}{1-(-2)} = \frac{-3}{3} = -1\]
        \[f[x_{1}, x_{2}, x_{3}, x_{4}] = \frac{f[x_{2}, x_{3}, x_{4}] - f[x_{1}, x_{2}, x_{3}]}{x_{4}-x_{1}} = \frac{-4-(-1)}{2-(-1)} = \frac{-3}{3} = -1\]
        \[f[x_{2}, x_{3}, x_{4}, x_{5}] = \frac{f[x_{3}, x_{4}, x_{5}] - f[x_{2}, x_{3}, x_{4}]}{x_{5}-x_{2}} = \frac{-7-(-4)}{3-0} = \frac{-3}{3} = -1\]
        The fourth divided differences are:
        \[f[x_{0}, x_{1}, x_{2}, x_{3}, x_{4}] = \frac{f[x_{1}, x_{2}, x_{3}, x_{4}] - f[x_{0}, x_{1}, x_{2}, x_{3}]}{x_{4}-x_{0}} = \frac{-1-(-1)}{2-(-2)} = \frac{0}{4} = 0\]
        \[f[x_{1}, x_{2}, x_{3}, x_{4}, x_{5}] = \frac{f[x_{2}, x_{3}, x_{4}, x_{5}] - f[x_{1}, x_{2}, x_{3}, x_{4}]}{x_{5}-x_{1}} = \frac{-1-(-1)}{3-(-1)} = \frac{0}{4} = 0\]

        %2b
        \item We will construct the Newton Interpolation formula
        \[ P(x) = f[x_{0}] + \sum_{k=1}^{n}f[x_{0}, x_{1}, ..., x_{k}](x-x_{1})(x-x_{2})... (x-x_{k-1})\]
        using the divided differences calculated in (a).
        \[P(x) = 1 + 3(x+2) + 2(x+2)(x+1) - 1(x+2)(x+1)x\]
        \[P(x) = 1 + 3x + 6 + 2x^{2}+6x + 4 - x^3 - 3x^2 - 2x\]
        \[P(x) = -x^3+(2x^2-3x^2)+ (3x+6x-2x)+(1+6+4)\]
        \[P(x) = -x^3-x^2+7x+11\].
    \end{enumerate}

    %Question 3
    \item
\subsection*{Cubic Spline Form}  
Each spline segment \( S_j(x) \) is a cubic polynomial of the form:
\[
S_j(x) = a_j + b_j (x - x_j) + c_j (x - x_j)^2 + d_j (x - x_j)^3, \quad x_j \leq x \leq x_{j+1}.
\]

\subsection*{Natural Cubic Spline Conditions}  
\begin{itemize}
    \item The spline must pass through each given point:
    \[
    S_j(x_j) = f(x_j), \quad S_j(x_{j+1}) = f(x_{j+1}).
    \]
   
    \item The first derivatives must be continuous at each interior point:
    \[
    S_j'(x_{j+1}) = S_{j+1}'(x_{j+1}).
    \]
    \item The second derivatives must also be continuous:
    \[
    S_j''(x_{j+1}) = S_{j+1}''(x_{j+1}).
    \]
    \item The second derivatives at the endpoints are zero:
    \[
    S_0''(x_0) = 0, \quad S_{n-1}''(x_n) = 0.
    \]
\end{itemize}

For simplicity, we set $h_j = x_{j+1} - x_j.$
Since the given points are equally spaced, then \( h_j \) is constant $(h_1=h_2=h_3=h_4 = 1)$.\\

\subsection*{System of Equations for \( c_j \)}  
Using the second derivative continuity condition and the natural spline assumption, we get a tridiagonal system:
\[
2(h_{j-1} + h_j)c_j + h_j c_{j+1} + h_{j-1} c_{j-1} = \frac{3}{h_j}(a_{j+1} - a_j) - \frac{3}{h_{j-1}}(a_j - a_{j-1}).
\]

Since $h_j = h_{j-1} = 1$, this simplifies to:
\[
3(a_{j+1} - a_j) - 3(a_j - a_{j-1}).
\]

%\newpage
For the interior points where the second derivative is continuous $(j=2,3,4)$:
For $j=2$
\[
c_1+4c_2+c_3 = 3(2-2(1)+3)=9.
\]
For $j=3$
\[
c_2+4c_3+c_4=3(3-2(2)+1)=3(0)=0.
\]
For $j=4$
\[
c_3+4c_4+c_5=3(2-2(3)+2)=3(-2)=-6.
\]
Since this is a natural cubic spline, $c_1=0,c_5=0.$\\
This simplifies the system of equations to:
\begin{align*}
4c_2 + c_3 &= 9, \\
c_2 + 4c_3 + c_4 &= 0, \\
c_3 + 4c_4 &= -6.
\end{align*}

\textbf{Now solving:}\\

For \( c_4 \):
\[
c_4 = -\frac{6 - c_3}{4}.
\]

Substituting into the second equation:

\[
c_2 + 4c_3 + \frac{-6 - c_3}{4} = 0.
\]

Multiplying by 4:

\[
4c_2 + 16c_3 - 6 - c_3 = 0.
\]

\[
4c_2 + 15c_3 = 6.
\]

For \( c_3 \):

\[
c_2 = \frac{6 - 15c_3}{4}.
\]

Substituting into the first equation:

\[
4\left(\frac{6 - 15c_3}{4}\right) + c_3 = 9.
\]

\[
6 - 15c_3 + c_3 = 9.
\]

\[
-14c_3 = 3.
\]

\[
c_3 = -\frac{3}{14}.
\]

Solve for \( c_4 \):

\[
c_4 = \frac{-6 - (-3/14)}{4} = \frac{-6 + 3/14}{4} = \frac{-84 + 3}{56} = \frac{-81}{56}.
\]

Solve for \( c_2 \):

\[
c_2 = \frac{6 - 15(-3/14)}{4} = \frac{6 + 45/14}{4} = \frac{84 + 45}{56} = \frac{129}{56}.
\]

\textbf{This yields the final values:}

\[
c_1 = 0, \quad c_2 = \frac{129}{56}, \quad c_3 = -\frac{3}{14}, \quad c_4 = -\frac{81}{56}, \quad c_5 = 0.
\]

    %Question 4
    \item
    \begin{enumerate}
        %4a
        \item
        %4b
        \item
    \end{enumerate}

    %Question 5 (Anton)
    \item
    \begin{enumerate}
        A natural cubic spline has second derivatives equal to \(0\) at the endpoints. The second derivative of a cubic 
        polynomial is a linear function. Let \(f(x)=ax^{3}+bx^{2}+cx+d\) be a general cubic polynomial, where \(a\ne 0\). 
        The first derivative is \(f^{\prime }(x)=3ax^{2}+2bx+c\). The second derivative is \(f^{\prime \prime }
        (x)=6ax+2b\). For \(f(x)\) to be its own natural cubic spline, \(f^{\prime \prime }(x_{0})=0\) and 
        \(f^{\prime\prime }(x_{1})=0\). This means \(6ax_{0}+2b=0\) and \(6ax_{1}+2b=0\). Subtracting the two equations, we 
        get \(6a(x_{1}-x_{0})=0\). Since \(x_{1}\ne x_{0}\), it must be that \(a=0\). If \(a=0\), then \(2b=0\), so 
        \(b=0\). This contradicts the assumption that \(f(x)\) is a cubic polynomial (\(a\ne 0\)). Therefore, \(f(x)\) 
        cannot be its own natural cubic spline unless \(f^{\prime \prime }(x)\) is identically zero, which means \(f(x)\) 
        is at most a linear function.
    \end{enumerate}
    
    %Question 6 (Joshua)
    \item
    \begin{enumerate}
        %6a
        \item
        \begin{proof}
            Note that by the triangle inequality, we have
            \[|m_{kk}x_k|=\left| \sum_{j\neq k} m_{kj}x_{j}\right|\leq \sum_{j\neq k} |m_{kj}||x_j|.\]
            By definition, $k$ was chosen such that $|x_{k}|= \max\limits_{1\leq j\leq n} |x_j|$. Therefore, since $|m_{kj}|$ is non-negative for all $j\neq k$, we must have
            \[|m_{kk}||x_k|=|m_{kk}x_k|\leq \sum_{j\neq k} |m_{kj}||x_j| \leq \sum_{j\neq k} |m_{kj}||x_k|=|x_k|\sum_{j\neq k} |m_{kj}|.\]
            Dividing by $|x_k|$ gives
            \[|m_{kk}|\leq \sum_{j\neq k} |m_{kj}|\]
            which is the desired result.
        \end{proof}
        %6b
        \item
        We may first assume that $h_i>0$ for all $i$, since we may simply choose the nodes $x_i$ in increasing order\footnote{Also note that we assume each node is distinct, therefore, $h_{i}\neq 0$ for all i}.
        Note that the matrix given by
        \[A_{ij}=\begin{cases}
            2(h_i+h_{i+1}) & \text{ if } i=j\\
            h_i & \text{ if } j = i-1 \text{ or } j= i+1\\
            0 & \text{otherwise}
        \end{cases}\]

        satisfies the following for each $i\neq 1$ and $i\neq n-1$:
        \[|A_{ii}|=|2(h_{i}+h_{i+1})|=2(h_{i}+h_{i+1})>2h_{i}=\sum_{j\neq i} A_{ij}.\]
        If $i=1$ or $i=n-1$, then we write that $2h_{i}>h_{i}=\sum_{j\neq i} A_{ij}$.
        Therefore, $A$ is strictly diagonally dominant, and thus invertible. 

        \bigskip

        Furthermore, note that since $a_j=f(x_{j})$, the constant coefficients, $a_j$ are uniquely determined\footnote{That is to say, for the family $\{S_j\}_{0\leq j \leq n-1}$ to be a cubic spline on $f$, the conditions derived in class must be satisfied. Showing that the coefficients which satisfy these conditions are unique, then shows that the cubic spline is unique. This will be assumed from now on.} by $f$. Since $A$ is invertible, there is a unique solution to the system
        \[A\mathbf{x}=\mathbf{b}\]
        where
        \begin{align*}
            \mathbf{x} = \begin{pmatrix}
                c_1 \\ 
                c_2 \\ 
                \vdots \\
                c_{n-1}\\
            \end{pmatrix} & & \text{and} && \mathbf{b}=
            \begin{pmatrix}
                \frac{3}{h_{1}}(a_{2}-a_{1}) - \frac{3}{h_{0}}(a_{1}-a_{0})\\
                \frac{3}{h_{2}}(a_{3}-a_{2}) - \frac{3}{h_{1}}(a_{2}-a_{1})\\ 
                \vdots\\
                \frac{3}{h_{n-1}}(a_{n}-a_{n-1})- \frac{3}{h_{n-2}}(a_{n-1}-a_{n-2})
            \end{pmatrix}.
        \end{align*}
        Therefore, since $a_j$ is uniquely determined by $f$, so is\footnote{We also note that $c_0=c_n=0$, so they are also unique.} $c_j$, and by extension 
        \[b_j:= \frac{1}{h_{j}}(a_{j+1}-a_{j})-\frac{h_{j}}{3}(c_{j+1}+2c_{})\]
        and 
        \[d_j:=\frac{1}{3h_{j}}(c_{j+1}-c_{j})\]
        are also uniquely determined by $f$. Therefore, the family $\{S_{j}\}_{0\leq j\leq n-1}$ is the unique cubic spline on $f$, which is the desired result.
    \end{enumerate}
\end{enumerate}
\end{document}
